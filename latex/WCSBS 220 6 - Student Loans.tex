%% LyX 2.3.6.2 created this file.  For more info, see http://www.lyx.org/.
%% Do not edit unless you really know what you are doing.
\documentclass[oneside,english]{amsart}
\usepackage[T1]{fontenc}
\usepackage{geometry}
\geometry{verbose,tmargin=1in,bmargin=1in,lmargin=1in,rmargin=1in}
\usepackage{babel}
\usepackage{textcomp}
\usepackage{url}
\usepackage{amstext}
\usepackage{amsthm}
\usepackage{graphicx}
\usepackage[unicode=true]
 {hyperref}

\makeatletter

%%%%%%%%%%%%%%%%%%%%%%%%%%%%%% LyX specific LaTeX commands.
%% Because html converters don't know tabularnewline
\providecommand{\tabularnewline}{\\}

%%%%%%%%%%%%%%%%%%%%%%%%%%%%%% Textclass specific LaTeX commands.
\numberwithin{equation}{section}
\numberwithin{figure}{section}
\theoremstyle{plain}
\newtheorem{thm}{\protect\theoremname}
\theoremstyle{definition}
\newtheorem{xca}[thm]{\protect\exercisename}
\theoremstyle{definition}
\newtheorem{defn}[thm]{\protect\definitionname}
\theoremstyle{definition}
\newtheorem{example}[thm]{\protect\examplename}

\makeatother

\providecommand{\definitionname}{Definition}
\providecommand{\examplename}{Example}
\providecommand{\exercisename}{Exercise}
\providecommand{\theoremname}{Theorem}

\begin{document}
\title{WCSBS 6 - Student Loans, Regression, \& Interest}
\maketitle

\section{Why Care?}
\begin{itemize}
\item $66\%$ of fourth-year college students in the US have an average
of $\$26,600$ in student loans.
\item {[}slide{]} The total U.S. student debt currently stands at \$$1.5$
trillion\textemdash the second-highest level of consumer debt behind
mortgages, and higher than both credit cards and auto loans.
\item The average debt for a $2017$ graduate is $\$28,650$.
\item Student loan debt is a social justice issue.
\begin{itemize}
\item About half of all individuals who default on their student loans never
earned a college credential. (elitism)
\item the typical Black or African American borrower is more than four times
as likely to default on student loans that their white peers. (racism)
\item about $27\%$ of those who default on student loans are parents. (classism)
\item more than $80\%$ of Pell Grant recipients come from families who
earn $\$40,000$ annually or less. Roughly $90\%$ of individuals
who default within $12$ years of enrolling in college received a
Pell Grant at some point. (classism)
\end{itemize}
\item {[}slide{]} student loan debt is in the news lately, especially due
to the Congressional hearing on 9/10/19 and Elizabeth Warren's proposal
to cancel student loan debt for most debtors.
\begin{itemize}
\item {[}slide{]} AoC making a student loan payment while at the Congressional
hearing
\end{itemize}
\item \textbf{Goal: }gain experience in applying your mathematical knowledge
to analyze and better understand your own financial situation and
those of your fellow students.
\begin{itemize}
\item examine the relationship between the year and cost of college tuition/fees
\end{itemize}
\end{itemize}

\section{Video response: Patriot Act on student loans}
\begin{enumerate}
\item What are some of the consequences that are faced by people who can't
pay off their student loans?
\begin{itemize}
\item aggressive debt collection practices: 
\begin{itemize}
\item suspending licenses of healthcare workers
\item being arrested by US Marshal's office for $\$1,500$ in debt (no legal
representation)
\end{itemize}
\end{itemize}
\item What happened in 2010 to change the landscape of student loan debt
in the US?
\begin{enumerate}
\item Government used to partner with banks, and government would be on
the hook for defaults instead of banks
\item Obama in $2010$ ``cut out the middlemen'' (banks) and students
began borrowing directly from the government.
\item Department of Education became the largest loan-giving bank in the
US.
\end{enumerate}
\item What are loan servicers, and what has their role been in the student
loan crisis? 
\begin{enumerate}
\item Companies that collect money, make sure you're paying on time, and
explain all your repayment options (theoretically)
\item Poor loan servicing causes problems for students. ``Financial terrorism''
\end{enumerate}
\item What are three ways in which Navient has mistreated borrowers? 
\begin{enumerate}
\item Overcharging troops
\item Damaging the credit of disabled borrowers, including double-charging
them
\item Bad customer service
\begin{enumerate}
\item Telling people they might have to live in their truck
\item Service was inadequate, including keeping calls to seven minutes or
under
\item March Madness-style bracket to have agents compete to speed through
as many clients as possible
\item Misleading borrowers by telling them to go in forbearance instead
of income-based repayment plans
\end{enumerate}
\end{enumerate}
\item What is an income-based repayment plan? What is forbearance, and why
has it negatively affected borrowers when Navient encourages borrowers
to enter forbearance instead of income-based repayment? 
\begin{enumerate}
\item Income-based repayment plans: allow people to tailor payments to their
income
\item Forbearance: stop making payments temporarily, but pay more interest
later
\begin{enumerate}
\item Has added $\$4$ billion in interest (according to Consumer Financial
Protection Bureau)
\end{enumerate}
\item Forbearance often isn't in the best interest of people, but it's faster
on the phone
\end{enumerate}
\item In what ways does Minhaj argue that the Department of Education has
\textquotedbl failed borrowers\textquotedbl ?
\begin{enumerate}
\item They keep hiring shitty loan servicers
\item They're not protecting borrowers
\item They mismanage loan forgiveness programs (e.g. by not actually processing
applications for people who go into public service and meet the forgiveness
requirements; only $96$ out of $30,000$ applications were approved
in $2017$.)
\end{enumerate}
\end{enumerate}
\begin{xca}
\textbf{(Reading Question) }For these questions, use the Excel spreadsheet
of college costs over time provided by the College Board. You\textquoteright ll
likely want to download the spreadsheet, since we'll use it for graphing
in Excel in class.
\begin{enumerate}
\item Use the spreadsheet to compare your current tuition with what your
peers paid twenty years ago. 
\item After looking at the spreadsheet carefully, what questions do you
have about college tuition? Write down two or three. 
\item What kind of graph do you think would work well for this data? Explain
your answer. 
\begin{enumerate}
\item You probably chose a \emph{line graph} (has time on $x$-axis and
some numerical variable on the $y$-axis) or a \emph{scatterplot}
(a numerical variable {[}years since $1978$-$79${]} on the $x$-axis
and a different one on the $y$-axis) for your graph.
\end{enumerate}
\item Try to graph the data presented in the spreadsheet. You can use Excel,
Google Sheets, or any other technological tool that can plot and fit
data points. Make separate graphs for private nonprofit four-year
colleges, public four-year colleges, and public two-year colleges,
starting from the academic year 1978-79 as year zero and going in
five-year intervals. In Excel, you just select the data you want to
graph, go up to the Ribbon/toolbar at the top of the screen, and click
Insert -> Recommended Charts. Try a few out; which ones work well
for you data and which don't? We'll discuss good graphs for this data
in class. Bring your answers to each question and the Excel/Google
document with your data/graphs to class on the date below for discussion.
\item What does ``enrollment-weighted'' mean?
\end{enumerate}
\end{xca}


\section{Data basics}
\begin{itemize}
\item What are data? \emph{Data} (plural) are careful observations made
as part of the scientific process.

\begin{itemize}
\item ex. the observation of whether the drug treatments eliminated kidney
stones
\end{itemize}
\item \emph{Statistics} is the study of how best to collect, analyze, and
draw conclusions from data.
\item The scientific process of investigation goes something like this:

\begin{enumerate}
\item Identify a question or problem.
\item Collect relevant data.
\item Analyze the data.
\item Form a conclusion.
\end{enumerate}
\item Statistics is interested in making stages 2-4 as objective as possible
(though we can't ever attain perfect objectivity!), rigorous, and
efficient.
\end{itemize}

\section{Reading Question: Going to College}
\begin{xca}
Please read Ch. 3 (Arms Race) of Weapons of Math Destruction, by the
data scientist Cathy O'Neil, then give the following questions your
best attempt. Bring your answers to class for discussion.
\end{xca}

\begin{enumerate}
\item Name three reasons that O'Neil considers the U.S. News \& World Report
college rankings a WMD. 
\begin{itemize}
\item Measuring ``educational excellence'', which isn't well-defined
\item the rankings are self-reinforcing: a bad ranking could prevent top
students/faculty from joining a college
\item the rankings have great scale: they are incredibly popular and widely-read,
squishing individual colleges into a single model
\end{itemize}
\item How did the U.S. News rankings change colleges? How did it change
access to college for students? How did it change tuition and fee
costs? 
\begin{enumerate}
\item Colleges began to game the rankings, sending false data on SAT scores,
acceptance/graduation rates, freshman retention, student-faculty ratio,
and alumni giving.
\item Colleges worked hard to improve their rankings because they didn't
have a choice. Because the rankings graded on a curve, it led to an
``arms race''.
\item Schools are all more selective to game the rankings, making the concept
of a ``safety school'' largely extinct.
\item Tuition and fees were left out of the US News model, giving college
presidents a ``gilded checkbook'' with which to game the rankings
by raising tuition and fees.
\end{enumerate}
\item At one point the author suggests that gaming the US News ranking might
not be bad for a university, as \textquotedblleft most of the proxies\dots{}
reflect a school\textquoteright s overall quality to some degree\textquotedblright{}
(58). Do you agree? 
\item If creating or running a WMD is so profitable, how can we push back
against them? 
\item Do you find other university ranking schemes to be preferable to the
US News one, either personally or within this book\textquoteright s
argument?
\end{enumerate}

\section{1.3 Overview of data collection principles}
\begin{itemize}
\item The first step in conducting research is to identify what you want
to investigate. Make it clear! 
\end{itemize}
\begin{xca}
Think about each of the following questions, then share in your groups: 
\begin{itemize}
\item What are some research questions you may want to investigate about
student loans?
\item What subjects or \textbf{cases }should be studied in each? (This is
called the \textbf{target population}.) 
\item What \textbf{variables} are important? 
\item It's usually too expensive to collect data for every case in a population.
How could you take a \textbf{sample}, or subset of the cases, which
is representative of the population? 
\end{itemize}
\end{xca}

\begin{itemize}
\item Tell students anecdotal evidence for each. 
\end{itemize}
\begin{defn}
\textbf{Anecdotal evidence }is evidence from particular situations
that we've heard about or experienced and remembered.
\begin{itemize}
\item Usually unusual cases that stick in our minds because they're striking 
\begin{itemize}
\item e.g. more likely to remember people who took 7 years to graduate 
\end{itemize}
\end{itemize}
\end{defn}

\begin{itemize}
\item How do we take a sample that's representative of a population? 
\begin{itemize}
\item Let's say we're trying to estimate the average student loan debt for
Westminster undergrads currently at the institution. What sample population
would you choose? 
\end{itemize}
\begin{enumerate}
\item Random selection 
\begin{itemize}
\item e.g. a raffle 
\item called a simple random sample 
\item Problems: what if only 30\% of people respond? 
\begin{itemize}
\item Maybe those people have the most free time because they're effective
at time management and will graduate earlier 
\item Maybe those people have free time because they're procrastinating
and will graduate later 
\item High non-response introduces non-response bias if respondents are
not representative 
\end{itemize}
\end{itemize}
\item Pick by hand, e.g. people you know 
\begin{itemize}
\item What major are most of your friends? 
\item What age are most of your friends? 
\item How might these influence the study? 
\item Bias is the skewing of a sample due to human error. 
\end{itemize}
\item Stop people walking around near Dick 
\begin{itemize}
\item More likely to get STEM majors 
\item Convenience sample is not representative 
\end{itemize}
\item Literally just ask every Westminster student, and spend lots of time
hunting them down
\begin{enumerate}
\item Census
\end{enumerate}
\end{enumerate}
\begin{itemize}
\item If 50\% of the ratings for a product on Amazon are negative, does
that mean 50\% of buyers are dissatisfied with the product? 
\end{itemize}
\item Explanatory and response variables 
\begin{itemize}
\item Let's say our question is: \textquotedblleft is federal spending,
on average, higher or lower in counties with high rates of poverty?\textquotedblright{} 
\begin{itemize}
\item if we suspect poverty affects spending in a county, then poverty is
the explanatory variable and spending is response variable 
\item what if we suspect counties are poor because the government spends
less in those counties? 
\item Doesn't actually mean poverty causes spending or vice versa; \textquotedblleft explanatory\textquotedblright{}
just reflects our guess 
\end{itemize}
\item What if our question is \textquotedblleft if homeownership is lower
than the national average, will the percent of multi-unit structures
be above or below average?\textquotedblright{} 
\begin{itemize}
\item Can't tell which is explanatory and which is response 
\end{itemize}
\end{itemize}
\end{itemize}
\rule[0.5ex]{1\columnwidth}{1pt}

{[}start 9-25-19{]}

\subsection{In the news}
\begin{itemize}
\item Harvard legacy/athlete preferences at Harvard {[}v.gd/EImanc{]}
\item How to measure who supports impeaching Trump after he attempted to
work with the Ukraine government to get dirt on Biden's son? {[}https://v.gd/3wyRf7{]}
\item Two types of data collection 
\begin{enumerate}
\item Observational studies 
\begin{enumerate}
\item collecting data in a way that does not interfere with the situation 
\begin{enumerate}
\item any examples? 
\item e.g. surveys, records, observing what happens 
\item correlation does not prove causation 
\end{enumerate}
\end{enumerate}
\item Experiments 
\begin{enumerate}
\item used to investigate causation 
\item examples? e.g. drug tests, stents 
\end{enumerate}
\end{enumerate}
\end{itemize}
\begin{xca}
If we want to measure student loan debt across all colleges and university
students in the US, should we use an experiment or an observational
study?

Go to the end of the College Board report, and try to figure out what
sampling strategy and data-collection methods they used.
\begin{itemize}
\item There are more than $5300$ colleges/universities, but they only measured
$4000$. {[}https://professionals.collegeboard.org/higher-ed/recruitment/annual-survey{]}
\item It was a Web survey and they only collected data from those institutions
that responded: convenience sample! But it got such a large proportion
of colleges/universities in the US to respond that we can at least
try to draw some conclusions from it.
\begin{itemize}
\item What traits may be true of those colleges/universities that responded
vs. failed to respond to the survey?
\end{itemize}
\end{itemize}
\end{xca}

~
\begin{xca}
Fill out the anonymous form at \url{https://tinyurl.com/WMGPA}. We'll
gather data in a second. (Pull up data and scatterplot it.) Then TPS: 
\begin{enumerate}
\item What is the explanatory and what is the response variable? 
\item Describe the relationship between the two variables. Make sure to
discuss unusual observations, if any. 
\item Is this an experiment or an observational study? 
\item Can we conclude that studying longer hours leads to higher GPAs? 
\end{enumerate}
\end{xca}


\section{Scatterplots}
\begin{itemize}
\item If we think one numerical/quantitative variable influences another,
(year and college tuition/fees, SAT scores and income, incidence of
autism and percent of population vaccinated, time spent listening
to Mozart in the womb and intelligence), then especially if we think
there's a pattern in their relationship (as income increases, so will
SAT scores), it can be useful to view the data in a \textbf{scatterplot. }
\end{itemize}
\begin{defn}
In a \textbf{scatterplot, t}he values of one variable appear on the
horizontal axis, and the values of the other variable appear on the
vertical axis. Each data point appears as the point $(x,y)$, where
$x$ is the $x$-axis value and $y$ is the $y$-axis value.
\end{defn}

\begin{itemize}
\item You can describe the overall pattern of a scatterplot by the \textbf{direction,
form, }and \textbf{strength }of the relationship.
\item The \textbf{correlation }is a number $R$ computed by a program which
falls between $-1$ and $1$. Positive $R$ means positive association;
negative $R$ means negative association.
\item Write in number line form:

\begin{tabular}{|c|c|c|}
\hline 
 & Positive & Negative\tabularnewline
\hline 
\hline 
Very & Strong positive correlation & Strong negative correlation\tabularnewline
\hline 
Slightly & Weak positive correlation & Weak negative correlation\tabularnewline
\hline 
\end{tabular}
\item The most common way to describe the relationship between two quantitative
variables is a scatterplot.
\end{itemize}

\subsubsection{Interpreting scatterplots}
\begin{itemize}
\item When examining a scatterplot, look for the overall pattern and any
striking deviations from that pattern.
\item You can describe the overall pattern of a scatterplot by the \textbf{direction,
form, }and \textbf{strength }of the relationship.
\item The \textbf{direction }answers the question: as the $x$-axis variable
increases, what happens to the $y$-axis variable?
\begin{itemize}
\item Two variables are \textbf{positively associated }when above-average
values of one tend to accompany above-average values of the other
and below-average values also tend to occur together. The scatterplot
slopes upward as we move from left to right.
\item Two variables are \textbf{negatively associated }when above-average
values of one tend to accompany below-average values of the other,
and vice versa. The scatterplot slopes downward from left to right.
\end{itemize}
\item The \textbf{form }is the overall shape of the graph outlined by the
data points in a scatterplot. 
\begin{itemize}
\item Is it a straight-line trend?
\item Is it a curved relationship?
\end{itemize}
\item The \textbf{strength }is determined by how closely the points follow
a clear form.
\item An important kind of deviation is an \textbf{outlier}, an individual
value which falls outside the overall pattern of the relationship.
\end{itemize}
\begin{xca}
{[}https://tinyurl.com/LifevsGDP{]} This figure is a scatterplot of
data from the World Bank. The individuals are all the world\textquoteright s
nations for which data are available. 
\begin{enumerate}
\item What are the explanatory and response variables in this situation?
\begin{itemize}
\item The explanatory variable is a measure of how rich a country is: the
gross domestic product (GDP) per person. GDP is the total value of
the goods and services produced in a country, converted into dollars. 
\item The response variable is life expectancy at birth.
\end{itemize}
\item What overall trend do we notice? What is the direction, form, and
strength of this relationship? Interpret this trend in context.
\begin{itemize}
\item We expect people in richer countries to live longer. The overall pattern
of the scatterplot does show this, but the relationship has an interesting
shape. Life expectancy tends to rise very quickly as GDP increases,
then levels off. 
\item People in very rich countries such as the United States typically
live no longer than people in poorer but not extremely poor nations.
Some of these countries, such as Costa Rica, even do better than the
United States.
\item The association is positive and very strong.
\end{itemize}
\item What outliers do you notice? Interpret these outliers in context.
\begin{itemize}
\item Three African nations are outliers. Their life expectancies are similar
to those of their neighbors but their GDPs are higher.
\item Equatorial Guinea and Gabon produce oil, and Sierra Leone produces
diamonds. It may be that income from mineral exports goes mainly to
a few people and so pulls up GDP per person without much effect on
either the income or the life expectancy of ordinary citizens. That
is, GDP per person is a mean, and we know that mean income can be
much higher than median income.
\end{itemize}
\end{enumerate}
\end{xca}

~
\begin{xca}
\textbf{14.1 Brain size and intelligence}. For centuries people have
associated intelligence with brain size. A recent study used magnetic
resonance imaging to measure the brain size of several individuals.
The IQ and brain size (in units of 10,000 pixels) of six individuals
are as follows: {[}slide{]}

Is there a clear explanatory variable? If so, what is it and what
is the response variable? Make a scatterplot of these data, then interpret
it. What is the form, direction, and strength of the association?
Are there any outliers?
\begin{itemize}
\item Either brain size or IQ could be the explanatory variable, since the
causality is unclear (and probably doesn't exist)
\begin{itemize}
\item If we switched which variables we thought of as explanatory and response,
our correlation wouldn't change, even though the slope of the best-fit
line did.
\end{itemize}
\item There's a very weak positive, straight-line correlation.
\begin{itemize}
\item Can we conclude that brain size affects intelligence based on this
data?
\begin{itemize}
\item No; it's only $6$ people!
\item Also, the correlation is way too weak.
\end{itemize}
\end{itemize}
\end{itemize}
\end{xca}


\section{Regression lines}
\begin{itemize}
\item If a scatterplot shows a straight-line relationship between two variables,
we'd like to summarize this overall pattern by drawing a line on the
graph.
\item A \textbf{regression line} is a straight line that describes how a
response variable $y$ changes as an explanatory variable $x$ changes. 
\begin{itemize}
\item We often use a regression line to predict values of $y$ from $x$.
\end{itemize}
\end{itemize}
\begin{xca}
{[}slide{]} The lengths of two bones in fossils of the extinct beast
archaeopteryx closely follow a straight-line pattern. {[}This is basically
saying leg length predicts wing length.{]} The existing data is from
$5$ fossils. If we find another incomplete fossil with femur length
$50$ cm and missing humerus, can we predict how long the humerus
is? Why or why not? How accurate do you think this prediction will
be?
\begin{itemize}
\item The straight-line pattern connecting humerus length to femur length
is so strong that we feel quite safe in using femur length to predict
humerus length. 
\item First, draw a line that fits the data well. Do this by hand now; we'll
learn how to do it better.
\item {[}slide{]} Then, starting at the femur length (50 cm), go up to the
line, then over to the humerus length axis. We predict a length of
about 56 cm. 
\item This is the length the humerus would have if this fossil\textquoteright s
point lay exactly on the line. All the other points are close to the
line, so we think the missing point would also be close to the line.
That is, we think this prediction will be quite accurate.
\end{itemize}
\end{xca}

~
\begin{xca}
{[}slide{]} The scatterplot shows the \% vote for the Democratic candidate
for President by state in $1980$ and $1984$. {[}The outlier is Georgia,
who voted for hometown Jimmy Carter in $1980$ and went more strongly
Reagan in $1984$.{]} Given a state that voted $40\%$ Democratic
in $1980$, can we predict the Democratic vote in that state in $1984$?
Why or why not? How accurate do you think this prediction will be?
\begin{itemize}
\item We could use the regression line drawn in Figure 15.2 to predict a
state\textquoteright s 1984 vote from its 1980 vote. 
\item The points in this figure are more widely scattered about the line
than are the points in the fossil bone plot in the previous slide. 
\item The correlations, which measure the strength of the straight-line
relationships, are $r=0.994$ for Figure 15.1 and $r=0.704$ for Figure
15.2. 
\item The scatter of the points makes it clear that predictions of voting
will be generally less accurate than predictions of bone length.
\end{itemize}
\end{xca}

\begin{itemize}
\item Different people might draw different regression lines by eye.
\end{itemize}
\begin{xca}
How could we measure how well a regression line fits actual data?
\begin{itemize}
\item 7The measure of how well a line fits data is the distance between
the actual and predicted $y$-coordinates, which is hard to guesstimate
by eye.
\item In addition, positive and negative distances could cancel each other
out\textendash we need to make the values positive!
\item But we also want very large distances to have a disproportionate effect\textendash we
should square the sum of the distances!
\end{itemize}
\end{xca}

\begin{defn}
The \textbf{least-squares regression line }of $y$ on $x$ is the
line that makes the sum of the squares of the vertical distances of
the data points from the line as small as possible. 

In practice, a computer will spit out the equation for the best least-squares
regression line if you give it data.
\end{defn}

\begin{itemize}
\item The equation you get will be of form $y=mx+b$, where $m$ is the
\textbf{slope} of the line and $b$ is the \textbf{$y$-intercept}.
\begin{itemize}
\item The \textbf{slope }is the amount $y$ changes when $x$ increases
by one.
\item The \textbf{$y$-intercept }is the value of $y$ when $x=0$.
\end{itemize}
\end{itemize}
\begin{xca}
The equation of the least-squares line for humerus vs. femur length
in Archaeopteryx is
\[
\text{humerus length}=-3.66+(1.197\times\text{femur length}).
\]
\begin{enumerate}
\item What is the slope of this line? The $y$-intercept? What meaning,
if any, does each have in context?
\begin{itemize}
\item The slope is $1.197$, which says that humerus length increases roughly
$1.197$ cm for every $1$ cm increase in femur length.
\end{itemize}
\item What would be the predicted humerus length for a fossil with femur
$50$ cm long using this equation?
\begin{itemize}
\item Plug in $50$
\end{itemize}
\item Sketch this line on axes and extend it out to $200$ cm on the $x$
axis. Can we use this line to predict the femur length for a fossil
with humerus length $74$ cm? $121$ cm?
\end{enumerate}
\end{xca}


\subsection{Understanding prediction}
\begin{itemize}
\item In practice, we use several explanatory variables to predict a response.
\item As part of its admissions process, a college might use ACT Math and
Verbal scores and high-school grades in English, math, and science
($5$ explanatory variables) to predict first-year college grades.
\item All statistical methods of predicting a response share the following
basic properties of least-squares regression lines:
\begin{itemize}
\item Prediction is based on fitting some ``model'' to a set of data.
That model is not always a straight line.
\item {[}slide{]} Prediction works best when the model fits the data closely.
It'd be inaccurate (and \textbf{overfitting})\textbf{ }to attempt
to predict revenue from corporate tax rate using the Laffer curve
in the slide.
\item {[}slide{]} Prediction outside the range of available data is risky.
The ``regression to the moon'' prediction that women will run faster
than the speed of sound in $2600$ is clearly inaccurate. 
\begin{itemize}
\item Prediction outside the range of available data is referred to as \textbf{extrapolation}.
Beware of extrapolation!
\end{itemize}
\end{itemize}
\end{itemize}

\subsection{Correlation}
\begin{xca}
{[}slide{]} Which of these scatterplots has a stronger positive association?
{[}they're the same{]}
\end{xca}

\begin{itemize}
\item It's easy to fool the eye. We need a numerical measure of the strength
of an association.
\item Our overall strategy for any data analysis: use a numerical measure
to supplement a graph.
\end{itemize}
\begin{defn}
The \textbf{correlation }describers the direction and strength of
a straight-line relationship between two quantitative variables. Correlation
is ususally written as $R$.
\end{defn}

\begin{itemize}
\item In practice, we'll think of $R$ as the output of a computer program
when we ask it for the correlation.
\item Correlation measures the strength and direction of a straight-line
relationship.
\item Regression draws a line to describe the relationship.
\item {[}slide{]} In both cases, outliers have a significant effect!
\begin{itemize}
\item Be wary if your scatterplot shows strong outliers.
\end{itemize}
\end{itemize}
\begin{xca}
In this slide, Hawaii is a high outlier. The orange best-fit line
works if Hawaii is included, and the black dotted best-fit line works
if Hawaii is removed. How would the strength of the relationship between
max annual precip and max $24$-hour precip change if we removed Hawaii? 
\begin{itemize}
\item The correlation for all $50$ states is $R=0.510$.
\item If we leave out Hawaii, the correlation drops to $R=0.248$\textendash almost
disappearing!
\item Outliers can make there appear to be correlation when there's really
not.
\end{itemize}
\end{xca}

\begin{itemize}
\item The usefulness of a regression line for prediction depends on the
strength of the association.
\item The right measure of the usefulness of a regression line is $R^{2}$.
\end{itemize}
\begin{defn}
The \textbf{square of the correlation}, $R^{2}$, is the proportion
of the variation in the values of $y$ that is explained by the trendline.
\begin{itemize}
\item The idea is that when there is a straight-line relationship, some
of the variation in $y$ is accounted for by the fact that as $x$
changes it pulls $y$ along with it.
\end{itemize}
\end{defn}

\begin{itemize}
\item In reporting a regression, it is usual to give $R^{2}$ as a measure
of how successful the regression was in explaining the response. When
you see a correlation, square it to get a better feel for the strength
of the association. 
\begin{itemize}
\item Perfect correlation ($R=-1$ or $R=1$) means the points lie exactly
on a line. Then $R^{2}=1$ and all of the variation in one variable
is accounted for by the straight-line relationship with the other
variable. 
\item If $R=-0.7$ or $R=0.7$, $R^{2}=0.49$ and about half the variation
is accounted for by the straight-line relationship. In the $R^{2}$
scale, correlation \textpm 0.7 is about halfway between 0 and \textpm 1.
\end{itemize}
\end{itemize}
\begin{xca}
{[}slide{]} There's lots of variation in the humerus length of these
fossils. Does it seem like a lot or a little of the variation in humerus
length is accounted for by variation in femur length? Given that $R=0.994$
for this relationship, what is the exact percentage of variation in
humerus length that's caused by variation in femur length? What does
this tell us about the accuracy of predicting humerus length using
femur length?
\begin{itemize}
\item There is a lot of variation in the humerus lengths of these 5 fossils,
from a low of 41 cm to a high of 84 cm. 
\item The scatterplot shows that we can explain almost all of this variation
by looking at femur length and at the regression line. As femur length
increases, it pulls humerus length up with it along the line. There
is very little leftover variation in humerus length, which appears
in the scatter of points about the line. 
\item Because $R=0.994$ for these data, $R^{2}=(0.994)^{2}=0.988$. So
the variation \textquotedblleft along the line\textquotedblright{}
as femur length pulls humerus length with it accounts for $98.8\%$
of all the variation in humerus length. The scatter of the points
about the line accounts for only the remaining 1.2\%. 
\item Little leftover scatter says that prediction will be accurate.
\end{itemize}
\end{xca}


\section{Week 5 homework: linear models \& climate change}
\begin{itemize}
\item In a week (October 7), the activist group Extinction Rebellion is
launching a week of direct, nonviolent action ``for climate and ecological
justice''. {[}rebellion.earth/international-rebellion{]}
\item What are the data supporting this rebellion? Against it? (Anthropocene
extinction, global warming, climate change, rainforests burning, Global
South and poor, PoC effects)
\end{itemize}

\section{Why Is College Tuition Twice as High as in $1995$?}
\begin{xca}
Student Loans Worksheet 1; use Desmos to compare your trendline functions
and to evaluate your trendline functions at various $x$-coordinates.
\end{xca}


\section{Student Loans 1a: reading questions}
\begin{xca}
\textbf{(HW) }The goal of the following assignment is to better understand
the complicated world of student loans, student loan debt, and the
politics of both.

The following four questions are about the College Board's Trends
in Student Aid publication from 2017.
\begin{enumerate}
\item Explain the difference between subsidized and unsubsidized federal
loans. 
\item When did the federal government begin giving unsubsidized loans to
students according to the chart in Trends in Student Aid 2015? 
\item Between the years 2000 and 2012, what was the trend for both subsidized
and unsubsidized loans? 
\item Define private student loans. 
\end{enumerate}
The following questions are about the article \textquotedbl The Student
Loan Debt Crisis in 9 Charts\textquotedbl{} by Mother Jones, accessible
online.
\begin{enumerate}
\item Choose five charts in the article and write a 200-to-300-word paragraph
explaining the meaning of the data presented in the graphs and how
the information presented in each are related to one another. 
\item In the figure that represents the percentage change of tuition and
The Consumer Price Index (CPI), explain the two trends and suggest
what type of functions (polynomial, exponential, linear, horizontal/vertical
lines, parabolas) can be used to fit each curve. Which trend is rising
faster? Give a mathematical argument. 
\item What is the size of the student loan debt? How does it compare with
the auto loan and the credit card industry? 
\item Compare the different scenarios for the total interest paid on a \$23,000
loan. In addition to turning your homework in online, please bring
a copy of it to class on the due date for discussion.
\end{enumerate}
\end{xca}


\section{Student Loans 2b: Compound Interest}
\begin{xca}
\textbf{{[}Desmos: tinyurl.com/RQCompound{]} (Reading Question) }Given
the following options for a student loan, which are you more likely
to choose and why? 
\begin{enumerate}
\item Borrowing \$10,000 at an interest rate of 5\%, compounded every year,
and paying the full amount owed ten years later. 
\begin{enumerate}
\item Balance in year $T$: $B_{1}(T)=\$10000(1.05)^{T}$
\item Amount paid: $B_{1}(10)=\$16288.95$
\end{enumerate}
\item Borrowing \$10,000 at an interest rate of 5\%, compounded every year,
and paying \$1000 per year until the loan is paid off. 
\begin{enumerate}
\item This is a complicated question, as is the next.
\item After year one, you'll owe $O(1)=10000(1.05)-1000$.
\item After year two, you'll owe $O(2)=[10000(1.05)-1000](1.05)-1000$.
\item This continues to change; does it ever equal zero? We don't quite
have the tools to analyze this situation currently.
\end{enumerate}
\item Borrowing \$10,000 at an interest rate of 5\%, compounded every year,
and paying \$2000 per year until the loan is paid off. 
\begin{enumerate}
\item Same deal, except this time you'll pay off your loan faster.
\end{enumerate}
\item Borrowing \$10,000 with \$100 interest every year and paying the full
amount owed ten years later. 
\item Borrowing \$10,000 with \$100 interest every year and paying \$1000
per year until the loan is paid off. 
\item Borrowing \$10,000 with \$100 interest every year and paying \$2000
per year until the loan is paid off. 
\end{enumerate}
\end{xca}


\subsection{Money Earns Money; Debt Loses Money}
\begin{itemize}
\item Imagine you take out a loan for $\$1,000$. Would you rather pay $\$100$
per year in interest, or $8\%$ per year in interest? In each case
the interest is taken from your bank account each year, and you aren't
paying off any of the loan.
\item The first scenario is called \textbf{simple interest}. You find the
new balance by adding $\$100$ each year to the previous balance. 
\item The second scenario is a little more complicated and is called \textbf{compound
interest}. The interest each year is a fixed percentage of the balance. 
\end{itemize}
\begin{example}
To find how much you owe after the first year, use the \textbf{one-plus
trick}: in other words, take the $8\%$ interest and add $1$ or $100\%$
to it to get $1.08$. Now multiply that $1.08$ by your principal,
$\$1,000$, to find the amount owed after year one:
\[
1.08\times\$1,000=\$1,080.
\]

After two years, you would again owe $8\%$ interest, but this time
on your balance of $\$1,080$. So after year two, you would owe
\[
1.08\times\$1,080=\$1,166.40.
\]
\end{example}

\rule[0.5ex]{1\columnwidth}{1pt}

{[}start here 10-7-19{]}
\begin{xca}
\textbf{{[}Reading Question: Student Loan Activism{]} }Roughly 70
percent of grads leave college with student debt, and over 44 million
Americans hold a total of~\$1.4 trillion in student loan debt. For
these reasons, student debt holders have the potential to be a powerful
political force. Organizations such as Freedom to Prosper are working
to politically organize student debt holders and other Americans in
favor of student debt cancellation by the government.
\begin{enumerate}
\item Using Freedom to Prosper's Debt Burden Index, research your home state
and current Congressional district (you can look up your district
by zip code here). What is the student debt burden score for your
home state/Congressional district? How does it compare to the Utah
and UT-02 (Utah's Second Congressional District, which includes Westminster
and Sugar House) averages? 
\begin{enumerate}
\item In my home state of Texas, the debt burden index is $55$. This is
lower than Utah's index of $67$ (surprising to me because of the
relatively low cost of Utah colleges).
\item In my home Congressional district, Texas's Second Congressional District,
the debt burden index is $58$. This is slightly less than UT-02's
score of $60$.
\end{enumerate}
\item At the bottom of this page, read about the methodology Freedom to
Prosper used to compute their debt burden indices. What do you like
about their methodology? What do you dislike? 
\begin{enumerate}
\item Like: attempts to cross-reference with income and racial demographcs,
uses well-gathered IRS data sets
\item Dislike: as the page mentions, the Statistics of Income data source
is a measure of tax returns, not individual debt. Low earners may
live in wealthy areas.
\item While the Statistics of Income data is the most comprehensive data
on student debt at the most granular geographic level available, it
is a measure of tax returns, not individuals. Some people file their
taxes jointly, and not everyone who is responsible for paying off
a student loan claims the deduction on their taxes. However, there
is a strong correlation between the share of student loan interest
deduction filings and the share of survey respondents who reporting
being responsible for student debt payments at various geographic
levels, so this does not seem to have systematically biased our results.
With respect to the construction of our burden index, adjusting the
rate of student loan deductions by other economic indicators imposes
a strong ecological assumption that doesn\textquoteright t hold in
all cases \textemdash{} there are certainly low-income student debt
holders who live in relatively wealthy areas. However, this adjustment
should, on average, move us closer to the accurate reflection of where
student debt burden is most acutely felt that would be available in
an ideal world where we had similar data at the individual level.
\end{enumerate}
\item Read Freedom to Prosper's \textquotedbl Truths about Student Debt
Holders\textquotedbl{} and pick one statistic that interests you.
Using estimation and/or Internet research, fact-check the statistic.
How true is the statistic? How misleading is Freedom to Prosper's
analysis of the statistic? 
\item In \textquotedbl Truth \#5\textquotedbl{} from the above page, Freedom
to Prosper presents statistics about Black and African-American student
loan debt. In what way(s) do you think historical oppression and current
racism impact student loan debt? In what way(s) does student loan
debt interact with and/or exacerbate racial inequalities?
\end{enumerate}
\end{xca}


\subsection{Student Loans Worksheet 2: Interest }
\begin{xca}
Imagine you take out a loan for $\$1,000$. Would you rather pay $\$100$
per year in interest, or $8\%$ per year in interest? In each case
the interest is taken from your bank account each year, and you aren't
paying off any of the loan.

The first scenario is called \textbf{simple interest}. You find the
new balance by adding $\$100$ each year to the previous balance. 

The second scenario is a little more complicated. The interest each
year is a fixed percentage of the balance. This scenario is called
\textbf{compound interest}. For example, after $1$ year, you'd owe
$8\%$ in interest on top of the $100\%$ you already owe, so you'd
owe $108\%=1.08$ times the initial balance of the loan:
\[
1.08\times\$1,000=\$1,080.
\]
\begin{enumerate}
\item Make a table of what you would owe under both scenarios now and in
years $1,2,3,$ and $4$.
\item Try to find an equation for your balance, $S$, on the simple interest
loan after $T$ years. What kind of function is this?
\begin{itemize}
\item $S(T)=1000+100T$
\item Linear function
\end{itemize}
\item Try to find an equation for your balance, $C$, on the compound interest
loan after $T$ years. What kind of function is this?
\begin{itemize}
\item $C(T)=1000(1.08)^{T}$
\item Exponential function
\end{itemize}
\item Using the graphing calculator at \url{https://desmos.com/calculator},
input $S(T)=$ followed by your equation for the simple interest loan.
Change the screen zoom so you can see the general shape of the graph.
What is the shape of the graph? Does this match your answer to \#2?
\begin{enumerate}
\item See \url{https://www.desmos.com/calculator/1eh5vlumxs} or \url{tinyurl.com/WCSBSInterest}.
\end{enumerate}
\item Using the graphing calculator at \url{https://desmos.com/calculator},
input $C(T)=$ followed by your equation for the compound interest
loan. Change the screen zoom so you can see the general shape of the
graph. What is the shape of the graph? Does this match your answer
to \#3?
\item After about how many years will the balance on both types of loan
be equal? Explain your answer.
\begin{enumerate}
\item Other than $T=0$, the only other intersection point of the graphs
is $(6.532,1653.22)$. After about six and a half years on either
plan, you'll owe $\$1653.22$.
\end{enumerate}
\item In the simple interest loan, how many years will it take for you to
owe double what you owed initially? What about in the compound interest
loan?
\begin{enumerate}
\item Simple interest: we solve for $T$ in the equation $S(T)=2000$:
\begin{align*}
2000 & =2(1000)=1000+100T\\
1000 & =100T\\
T & =10\text{ years}.
\end{align*}
\item Compound interest: we solve for $T$ in the equation $C(T)=2000$:
\begin{align*}
1000(1.08)^{T} & =2000\\
(1.08)^{T} & =2\\
T & =\log_{1.08}(2)\approx9.006\text{ years}.
\end{align*}
\end{enumerate}
\item Which loan would you rather take out if you expect to pay off all
your debt at once in $5$ years? Explain your answer.
\begin{enumerate}
\item When $T=5$, the simple interest loan balance is (barely) higher:
$S(5)=1500$, while $C(5)=1469.328$. 
\item I'd rather take out the compound interest loan in this case.
\end{enumerate}
\item Which loan would you rather take out if you expect to pay off all
your debt at once in $10$ years? Explain your answer.
\begin{enumerate}
\item When $T=10$, the compound interest loan has a higher balance: $C(10)=2158.925$,
while $S(10)=2000$.
\item Therefore, I'd rather take out the simple interest loan in this case.
\end{enumerate}
\item Now, where exactly is the ``inflection point'' where the simple
interest loan becomes cheaper? If you wanted to take out a loan for
$8$ or $9$ years, which would be better? in your equations for $S(T)$
and $C(T)$, replace the number $1000$ with the letter $P$ (for
the \textbf{principal} amount of your loan), the number $100$ with
$m$ (for the slope of your function), and $1.08$ with $R$ (for
the interest \textbf{rate} of the compound interest loan). Desmos
should prompt you to ``add slider''; click on the prompt.
\begin{enumerate}
\item Change the value of $m$ to $100$, then slide the value of $m$ down.
What happens to the interest owed as $m$ decreases? How much would
you owe after $5$ years if $m=50$? If $m=500$? How does the shape
of the graph change when $m$ is negative? Then change the value of
$m$ back to $100$.
\begin{itemize}
\item {[}demonstrate{]} $m$ changes the slope of the line by rotating the
line around the $y$-intercept like a hinge.
\item When $m=50$, you'd owe $\$1250$.
\item When $m=500$, you'd owe $\$3500$
\item When $m<0$, the line has a negative slope, so the loan balance would
decrease instead of increasing over time.
\end{itemize}
\item How does the interest rate affect which loan would be cheaper? Change
the value of $R$ to $1.08$, then slide the value of $R$ around.
What happens to the interest owed as $R$ decreases? Increases? How
much would you owe after five years if $R=1.2$? If $R=1.02$? How
does the shape of the graph change when $R=1$? $R<1$? Then change
the value of $R$ back to $1.08$.
\begin{itemize}
\item {[}demonstrate{]} Changing $R$ drastically changes the curvature
of the compound interest graph. Small changes in interest rate lead
to large changes in amount owed!
\item As $R$ decreases, the balance on the loan decreases.
\item As $R$ increases, the balance on the loan increases.
\item If $R=1.2$, then $C(5)$ would equal $\$2488.32$.
\item If $R=1.02$, then $C(5)$ would equal $\$1104.08$.
\item When $R=1$, the graph is a horizontal line. No interest is accrued.
\item When $R<1$, the graph represents exponential decay and is decreasing.
\end{itemize}
\item How does the principal amount of the loan affect which loan would
be cheaper? Change the value of $P$ to $1000$, then slide the value
of $P$ around. What happens to the amount owed over time as $P$
decreases? Increases? How much would you owe after five years if $P=500$?
If $P=2000$? How does the shape of the graph change when $P=0$?
$P<0$? Is this what you would expect?
\begin{itemize}
\item {[}demonstrate{]} Increasing the value of $P$ shifts both $S(T)$
and $C(T)$ up and to the left.
\item {[}demonstrate{]} Decreasing the value of $P$ shifts both $S(T)$
and $C(T)$ down and to the right. Interestingly, the intersection
points between $S(T)$ and $C(T)$ also change, so whether a simple
or compound interest loan is better for you depends, among other things,
on the principal.
\item If $P=500$, $C(5)=\$734.66$.
\item If $P=2000$, $C(5)=\$2938.66$.
\item {[}demonstrate{]} as we shift $P$, the compound and simple interest
graphs both shift up and to the left (for increasing $P$) or down
and to the right (for decreasing $P$). However, the effect on the
compound vs. simple interest graphs are not equal. As $P$ changes,
the compound and simple interest graphs do a complicated dance that
makes it somewhat difficult to predict which would be a better option.
\end{itemize}
\end{enumerate}
\item If your loan provider offered to halve your interest, how much would
you save over $10$ years in the first (simple interest) scenario?
In the second (compound interest) scenario (where your rate is $4\%$
rather than $8\%$)?
\item Consult the following link to learn how the interest on student loans
is calculated: \url{https://studentaid.ed.gov/sa/types/loans/interest-rates#what-interest}.
Next, calculate and compare the interest owed on the following loans: 
\begin{itemize}
\item Loan A: loan with principal amount of $\$15000$ with interest at
$7\%$, assuming you made your last payment 30 days ago. 
\item Loan B: loan with principal amount of $\$15000$ with interest at
$4\%$, assuming you made your last payment 30 days ago.
\item Loan C: loan with principal amount of $\$15000$ with interest at
$7\%$, assuming you made your last payment 60 days ago.
\item Loan D: loan with principal amount of $\$15000$ with interest at
$7\%$, assuming you made your last payment 120 days ago. 
\end{itemize}
\end{enumerate}
\end{xca}


\section{Student Loans 3a: Repayment Plans}
\begin{xca}
Please read \href{https://www.nerdwallet.com/blog/loans/student-loans/student-loan-repayment-plans/?educalc}{this article}
from financial advice site Nerdwallet, then answer the following questions.
(If you don't have student loans, pretend you have loans amounting
to 50\% of your tuition costs and answer the questions accordingly).
Turn in your answers in a text box below and bring them to class on
the date below for discussion.
\begin{enumerate}
\item Do you anticipate being eligible for Public Service Loan Forgiveness
(PSLF), or is this option interesting to you? Why or why not? 
\begin{itemize}
\item \textbf{Choose an income-driven plan if public service forgiveness
is an option}: Public Service Loan Forgiveness is a federal program
available to government and nonprofit employees. If you\textquoteright re
eligible, you can get your remaining loan balance forgiven tax-free
after you make 120 qualifying loan payments. You need to make most
of those payments on a federal income-driven repayment plan to benefit
from PSLF. Otherwise, you\textquoteright ll end up paying off the
loan before you\textquoteright re eligible for forgiveness. 
\end{itemize}
\item Fill out the form entitled \textquotedbl Is an income-driven repayment
plan right for you?\textquotedbl{} and describe your results. Why
is(n't) it recommended for you to enroll in a repayment plan based
on your information? 
\begin{itemize}
\item Because you're nailing your payments, want to pay your loans off quickly,
and don't work for an organization that would qualify you for Public
Service Loan Forgiveness, an income-driven repayment plan is probably
not a good option for you. For borrowers with strong credit and income
profiles, refinancing student loans is another option for reducing
the monthly payment. 
\item Because you'd like to pay your loans off quickly, you may want to
avoid being on an income-driven repayment plan for a long period of
time, as these plans extend your repayment period. However, it could
still make sense for you to take advantage of an IDR plan for a shorter
period of time if you are having trouble making your monthly payments.
You can remain on an IDR plan and pay more than the minimum per month,
with no penalty, if you'd like to pay your loans off faster. If you
have questions about which specific plan is best for you, call your
student loan servicer or calculate your payments with the estimator
below. 
\end{itemize}
\item Compare and contrast the various income-driven repayment plans, including
Income-based repayment, Income-contingent repayment, Pay As You Earn,
and Revised Pay As You Earn. What are the similarities and differences
between each plan? Which do you think would work best for your loans
based on the information given? 

\includegraphics[scale=0.75,bb = 0 0 200 100, draft, type=eps]{pasted2.png}
\item Compare and contrast the basic federal repayment plans, including
standard, graduated, and extended repayment. What are the similarities
and differences between each plan? Which do you think would work best
for your loans based on the information given?

\includegraphics[scale=0.75,bb = 0 0 200 100, draft, type=eps]{pasted3.png}
\end{enumerate}
\end{xca}


\subsection{Student Loans Worksheet 3a: You and Giselda}
\begin{xca}
\textbf{{[}RQ: when do students pay off debt?{]} }Read this CNBC article
entitled \textquotedbl This Is the Age Most Americans Pay Off Student
Loans\textquotedbl , then answer the following questions in a text
box or file upload. Also bring a copy of your responses to class on
the date below for discussion.
\end{xca}

\begin{enumerate}
\item Describe the graph entitled \textquotedbl Number of Student Loan
Borrowers by Age Group\textquotedbl{} in words, including the direction,
form, and strength of any trends observed. What years are outliers,
and why might those years be outliers? 
\begin{enumerate}
\item The first graph shows that Americans of all age groups are borrowing
student loans at an increasing rate between $2004$ and $2015$.
\item There are strong, roughly linear, positive trends among all age groups,
with the least strong trend being the under-30 age group.
\item The main outlier is $2010$-$2012$ for Americans under $30$, when
students borrowed at a decreasing rate. This is likely due to the
Great Recession, when many younger Americans put off college due to
lack of funds.
\end{enumerate}
\item Do the same for the graph entitled \textquotedbl Total Student Loan
Balances by Age Group\textquotedbl . 
\begin{enumerate}
\item There are strong, positive, exponential trends in student loan balances
among all age groups except under-30.
\item There is a strong, positive, linear trend in student loan balances
among Americans under 30.
\item Outliers include the years $2011$-$2013$ for Americans under $30$,
when there was a drastic increase in the total student loan balances.
This could be due to the aftereffects of the Great Recession when
younger Americans weren't able to pay off existing loans, making their
balances increase more than average due to interest or default.
\end{enumerate}
\item Choose a statistic from the article and fact-check it, using either
estimation or the Internet. How true is the statistic? How misleading?
Explain your answer.
\end{enumerate}
\begin{xca}
\textbf{{[}Worksheet 3a: You and Giselda{]} }Imagine the following
scenario. If you have student loans, Giselda attended a four-year
private liberal arts college between 1999-2003 and borrowed the same
percentage of her tuition as you did. Suppose that both you and Giselda
will pay off your student loans within ten years. If you don\textquoteright t
have student loans (lucky!), pretend that you and Giselda both borrowed
50\% of your tuition costs, but Giselda attended a private four-year
liberal arts college between 1999-2003.

Use the table below to make a payment plan for both you and Giselda.
Try to make the table reflect your loan situation as closely as possible,
and consider various options for how long it will take to pay off
your loans. (Usually, of course, the optimal situation is to pay off
loans as quickly as possible, but for many American students, this
isn\textquoteright t feasible.) 

Use your models for college tuition in 1999 and your actual Westminster
tuition + fees, less any scholarships/grants, to fill in the Tuition
row of the table.
\begin{itemize}
\item Tuition and fees at four-year private nonprofit institutions like
Westminster can be modeled using the equation $N(T)=\$531.52T+\$7143.40$,
where $T$ is measured in years after the $1978$-$79$ academic year.
\item Since Giselda started in the $1999$-$2000$ academic year, $21$
years after the $1978$-$79$ academic year, her costs can be estimated
by plugging in $T=21$ to the equation above:
\[
N(21)=\$531.52(21)+\$7143.40=\$18,305.32.
\]
\item If you started college in $2018$-$2019$, $30$ years after the $1978$-$79$
academic year, your tuition and fees would be estimated as
\[
N(30)=\$23,089.
\]
\end{itemize}
\end{xca}


\subsection{Student Loans Worksheet 3b: You and Fatimah}

Imagine the following scenario. Fatimah attended a four-year private
liberal arts college at the same time as you and borrowed the same
percentage of their tuition as you did. (If you don\textquoteright t
have student loans, assume you and Fatimah both borrowed 50\% of tuition
costs.) Suppose that both you and Fatimah will pay off your student
loans within ten years, but your interest rate is double Fatimah\textquoteright s.
Use the table below to make a payment plan for both you and Fatimah.
Try to make the table reflect your loan situation as closely as possible,
and consider various options for how long it will take to pay off
your loans. (Usually, of course, the optimal situation is to pay off
loans as quickly as possible, but for many American students, this
isn\textquoteright t feasible.)
\end{document}
