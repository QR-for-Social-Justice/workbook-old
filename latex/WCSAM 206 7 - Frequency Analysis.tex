%% LyX 2.3.6.2 created this file.  For more info, see http://www.lyx.org/.
%% Do not edit unless you really know what you are doing.
\documentclass[oneside,english]{amsart}
\renewcommand{\familydefault}{\sfdefault}
\usepackage[T1]{fontenc}
\usepackage{geometry}
\geometry{verbose,tmargin=1in,bmargin=1in,lmargin=1in,rmargin=1in}
\usepackage{amsthm}
\usepackage{graphicx}

\makeatletter
%%%%%%%%%%%%%%%%%%%%%%%%%%%%%% Textclass specific LaTeX commands.
\numberwithin{equation}{section}
\numberwithin{figure}{section}
\theoremstyle{plain}
\newtheorem{thm}{\protect\theoremname}
\theoremstyle{definition}
\newtheorem{xca}[thm]{\protect\exercisename}
\theoremstyle{definition}
\newtheorem*{xca*}{\protect\exercisename}

\makeatother

\usepackage{babel}
\providecommand{\exercisename}{Exercise}
\providecommand{\theoremname}{Theorem}

\begin{document}
\title{WCSAM 206 Unit 7 - Frequency Analysis for Shift Ciphers}
\maketitle

\section*{Last Time}
\begin{itemize}
\item We discovered how to decrypt affine ciphers once you know the key.
\item It's a matter of solving the encryption equation $C\equiv mP+k\mod n$
for $P$:
\begin{enumerate}
\item Subtract $k$ from both sides: $C-k\equiv mP\mod n$.
\item Multiply both sides by $m^{-1}$, the multiplicative inverse of $m\mod n$:
\[
m^{-1}(C-k)\equiv P\mod n
\]
\end{enumerate}
\end{itemize}
\textbf{Today}: We'll learn how to break shift and affine ciphers
(and all monoalphabetic substitution ciphers) \textbf{without knowing
the key(s) to start with}. The tools we'll need come from the Arabs
around 900 A.D.
\begin{itemize}
\item These tools were so effective that they killed off the shift cipher
once they were common knowledge!
\item The extended Euclidean algorithm wasn't published until $1740$, so
unless people were able to guess-and-check multiplicative inverses,
the affine cipher was safe in comparison.
\end{itemize}

\section{Reading questions}
\begin{xca}
{[}slide{]} Read the Sherlock Holmes story \textquotedbl The Adventure
of the Dancing Men\textquotedbl{} only through the end of Ch. 11 by
clicking on the title. Since we've discussed frequency analysis, you
now have all the tools you need to crack the code before Sherlock
does! Don't read ahead until you've given your best attempt to cracking
the code, and bring your work to class on the date below for group
discussion.
\begin{itemize}
\item {[}slide{]} solution
\item have students read decrypted forms of messages
\end{itemize}
(pass ciphertext clockwise in groups) Read p14-17 of The Code Book,
until the last full paragraph, the quote from al-Kindi. Then make
up some short (\textasciitilde 10 letters) plaintext where the most
common letter is something unusual, like x or z. Encrypt your plaintext
with an affine cipher of your choice and bring your plaintext and
ciphertext, on separate sheets, to class on the date below.
\begin{itemize}
\item Note how hard it is with short messages! A computer could check all
the $26^{2}$ affine cipher possibilities pretty quickly, though.
\end{itemize}
\end{xca}


\section{The Arab cryptanalysts and breaking simple shift ciphers}

Read p14-17 (until the last full paragraph, the quote from al-Kindi).
al-Kindi's technique became known as \textbf{frequency analysis},
and it's surprisingly straightforward to apply. 

\subsection*{The Process of Frequency Analysis}

Flip to the table at the bottom of p19 of your books. There's a listing
of the most common letters used in English:

\includegraphics[scale=0.25]{\string"Old Cryptography Notes/1000px-English_letter_frequency_(frequency)\string".png} 

You'll want to refer back to this as you do the next exercise.

Note that this doesn't always apply\textendash some texts use letters
less frequently. For example, in the book \textbf{A Void} by Georges
Perec {[}write this phrase on board by itself somewhere{]}, the letter
``e'' is not used at all!

You might ask how reliable the listing of common letters is. Figure
1.2 shows a stacked bar chart that shows the frequencies of the letters
in ``The Gold-Bug'', the 2006 State of the Union address, ``Julius
Caesar'', and the ``USA Patriot Act'':

\includegraphics{\string"Old Cryptography Notes/pasted7\string".png}
\begin{xca}
{[}CoursePack p99{]} ``Breaking the Code.pdf''
\end{xca}


\section*{Solving Simple Substitution Ciphers}

\textbf{For group work}: designate one person in the group to write
down ideas as they're said so you remember them. Designate another
person to write out your solution at the end. Designate two people
to generate as many ideas as possible, and the other two people to
respond to those ideas.
\begin{itemize}
\item This was the experience of many cryptographers after al-Kindi's technique
of frequency analysis became commonly known.
\item What we've done in building codes was hard, and cracking them may
seem too easy given the painstaking work we've done to encrypt our
messages. 
\begin{itemize}
\item However, it's often true that codes are easier to break than to make.
\end{itemize}
\item Frequency analysis will crack both shift and affine ciphers and reveal
the weaknesses of monoalphabetic substitution ciphers of any kind.
\item What we've done so far has been hard, but you've been on top of it!
\item The next cipher we'll discuss will be a bit easier to encrypt than
the affine cipher.
\item However, frequency analysis isn't perfect:
\begin{itemize}
\item Issue: our standard English frequencies don't always apply. 
\item For example, in the book \textbf{A Void} by Georges Perec {[}write
this phrase on board by itself somewhere{]}, the letter ``e'' is
not used at all! 
\item But the longer the message, the more likely it is to conform to standard
English frequencies.
\end{itemize}
\end{itemize}
\begin{xca*}
{[}handout; slide{]} Assume we have intercepted this scrambled message.
The challenge is to decipher it. We know the text is in English, and
that it has been scrambled according to a monoalphabetic substitution
cipher with a keyphrase, but we don't know the key. Use frequency
analysis. {[}This is a hard problem\textendash work in your groups!{]}

\includegraphics{\string"Old Cryptography Notes/pasted5\string".png}
\end{xca*}
\textbf{Solution. }Frequency analysis gives the following table of
letters:

\includegraphics{\string"Old Cryptography Notes/pasted6\string".png}

The three most common letters are \textbf{O, X, }and \textbf{P.} We
focus on how they appear next to all the other letters. For example,
if \textbf{O }represents a vowel, it should appear before and after
most of the other letters, whereas if it represents a consonant, it
will tend to avoid many of the other letters.

\textbf{O }neighbors the majority of letters, and so does \textbf{X.}
But \textbf{P }avoids most other letters and clusters around only
a few of them. This suggests that \textbf{O }and \textbf{X }are vowels,
while \textbf{P }is a consonant.

Hence \textbf{O }and \textbf{X }are probably \textbf{e} and \textbf{a}.
But which is which? Notice that \textbf{OO} appears twice, while \textbf{XX}
doesn't ever appear. Since \textbf{ee} is more common in English than
\textbf{aa}, probably $O=e$ and $X=a$.

The other letters that appear on their own are \textbf{$X=a$ }(which
is expected since ``a'' is a word) and \textbf{Y.} Hence probably
\textbf{Y} is \textbf{i}, the only other one-letter English word.

The letter \textbf{h }often goes before the letter \textbf{e}, but
rarely after it. This is characteristic of \textbf{B,} so probably
$B=h$. Then start replacing letters. Guess where ``the'', ``and''
appear. Then \textbf{C} must be a vowel and $C=o$. Then $K=s$. Get
the phrase ``thoMsand and one niDhts''. Hence $M=u$ and $D=g$.
Now guess the keyphrase is ``A VOID BY GEORGES PEREC'', which is
reduced to ``AVOIDBYGERSPC'' after removing spaces and repetitions.

\section*{Reflection}
\begin{enumerate}
\item What were the two most important concepts from today's class?
\item What were two things you didn't fully understand?
\item What were your two favorite things we did in class today?
\end{enumerate}

\section*{Homework (start in-class if time)}

\includegraphics{\string"Old Cryptography Notes/pasted126\string".png}

\includegraphics{\string"Old Cryptography Notes/pasted99\string".png}

\includegraphics{\string"Old Cryptography Notes/pasted124\string".png}
\end{document}
