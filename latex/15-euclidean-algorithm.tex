%% LyX 2.3.6.2 created this file.  For more info, see http://www.lyx.org/.
%% Do not edit unless you really know what you are doing.
\documentclass[oneside,english]{amsart}
\usepackage[T1]{fontenc}
\usepackage{geometry}
\geometry{verbose,tmargin=1in,bmargin=1in,lmargin=1in,rmargin=1in}
\setcounter{secnumdepth}{2}
\setcounter{tocdepth}{2}
\usepackage{amstext}
\usepackage{amsthm}

\makeatletter

%%%%%%%%%%%%%%%%%%%%%%%%%%%%%% LyX specific LaTeX commands.
%% Because html converters don't know tabularnewline
\providecommand{\tabularnewline}{\\}

%%%%%%%%%%%%%%%%%%%%%%%%%%%%%% Textclass specific LaTeX commands.
\numberwithin{equation}{section}
\numberwithin{figure}{section}
\theoremstyle{plain}
\newtheorem{thm}{\protect\theoremname}
\theoremstyle{definition}
\newtheorem{xca}[thm]{\protect\exercisename}
\theoremstyle{definition}
\newtheorem{example}[thm]{\protect\examplename}
\theoremstyle{definition}
\newtheorem{defn}[thm]{\protect\definitionname}
\theoremstyle{plain}
\newtheorem{cor}[thm]{\protect\corollaryname}

\makeatother

\usepackage{babel}
\providecommand{\corollaryname}{Corollary}
\providecommand{\definitionname}{Definition}
\providecommand{\examplename}{Example}
\providecommand{\exercisename}{Exercise}
\providecommand{\theoremname}{Theorem}

\begin{document}
\title{WCSAM 206 5 - Modular Inverses \& the Euclidean Algorithm}
\maketitle

\section{Warmup}
\begin{xca}
Who had a chance to do the exercises on the {[}slide 1{]}?
\end{xca}

\begin{itemize}
\item \textbf{Recall}: the \textbf{\emph{affine cipher}} encrypts a plaintext
number $P$ using the formula 
\[
C\equiv mP+k\mod26,
\]
where $m$ is called the \textbf{multiplicative key }and $k$ the
\textbf{additive key}.
\item To encrypt a message using an affine cipher, it's often useful to
make a table as follows: 

\begin{tabular}{|c|c|c|c|c|c|}
\hline 
Plaintext letter & g & h & o & s & t\tabularnewline
\hline 
\hline 
Plaintext number $P$ & 6 & 7 & 14 & 18 & 19\tabularnewline
\hline 
Ciphertext number $C\equiv7P+1\mod26$ & $43\equiv17$ & $50\equiv24$ & ... &  & \tabularnewline
\hline 
Ciphertext letter & R & Y & ... &  & \tabularnewline
\hline 
\end{tabular}
\item When we pass off this work to computers, these four steps will be
what we'll tell the computer to do so that we don't have to encrypt
long affine messages by hand.
\end{itemize}
\begin{example}
Let's discuss encryption of ``assassinate the king now'' using the
affine cipher $C\equiv3P+1\mod26$.

\begin{tabular}{|c|c|c|c|c|c|c|c|c|c|c|c|}
\hline 
Plaintext letter & A & S & S & A & S & S & I & N & A & T & E\tabularnewline
\hline 
\hline 
Plaintext number & 0 & 18 & 18 & 0 & 18 & 18 & 8 & 13 & 0 & 19 & 4\tabularnewline
\hline 
$3\times$ plaintext number $+1$ (mod 26) & 1 & $-17\equiv9$ & 9 & 1 & 9 & 9 & 25 & $40\equiv14$ & 1 & $-20\equiv6$ & ...finish this yourselves.\tabularnewline
\hline 
Ciphertext letter & B &  &  &  &  &  &  &  &  &  & \tabularnewline
\hline 
Plaintext letter & T & H & E & K & I & N & G & N & O & W & \tabularnewline
\hline 
Plain number & 19 & 7 & 4 & 10 & 8 & 13 & 6 & 13 & 14 & 22 & \tabularnewline
\hline 
$3\times$ Plain number $+1$ &  &  &  &  &  &  &  &  &  &  & \tabularnewline
\hline 
Cipher  &  &  &  &  &  &  &  &  &  &  & \tabularnewline
\hline 
\end{tabular}
\end{example}

\begin{xca}
\textbf{{[}RQ{]} }Encrypt the following using the affine cipher $C\equiv mP+k\mod26$,
the standard alphabet, and the specified keys.
\begin{enumerate}
\item \textquotedbl do not erase\textquotedbl , $m=19,k=7$

mn une fshlf
\item \textquotedbl never say never\textquotedbl , $m=4,k=0$

aqgqq uas aqgqq

Notice plaintext n, v, and a are encrypted as A!
\item \textquotedbl supercalifragilisticexpialidocious\textquotedbl ,
$m=13,k=0$

Why would this give us only Ns and As as ciphertext?
\item Computers can only understand numbers (which are written in 0s and
1s), not letters. The computer language ASCII represents all commonly-used
written characters as numbers between 1 and 256. Do you think we can
encrypt ASCII messages using the affine cipher $C\equiv8P\mod256$?
Why or why not? 
\begin{enumerate}
\item Nope; $\gcd(8,256)=8>1$.
\end{enumerate}
\end{enumerate}
From the work you did above, what multiplicative keys do you think
\textquotedbl work\textquotedbl{} for affine ciphers? What is the
\textquotedbl problem\textquotedbl{} with the keys that don't work?
Even if you don't have a complete sense of the issue, explain your
hypothesis.
\end{xca}

\begin{itemize}
\item {[}slide{]} Why does this happen? Think back to your mod-12 multiplication
tables; what numbers had all integers $\mod12$ in their columns?
Which didn't?
\end{itemize}

\section{Warmup 2: multiplicative inverses $\mod26$}
\begin{example}
{[}slide{]} But now let's say that we're trying to crack the affine
cipher. Let's take this ciphertext: \textbf{hikctyhghykaezztgaayrhbggrvgdyfg}
and try to find $m$ and $k$.
\begin{itemize}
\item The most common letters are g and h, which appear six and four times,
respectively. This suggests that the ciphertext g and h correspond
to plaintext e and t. Let's write the previous statement in number
form:
\item Ciphertext 6 corresponds to plaintext 4, and ciphertext 7 corresponds
to plaintext 19.
\item Let's look at the equation $(\ast)$ above. When $c_{i}=6$, then
$p_{i}=4$ and the equation becomes
\[
6\equiv4m+k.
\]
When $c_{i}=7$, then $p_{i}=19$ and the equation becomes
\[
7\equiv19m+k.
\]
\item Together, these equations give us a \textbf{system of congruences}
\begin{eqnarray*}
4m+k & \equiv & 6\\
19m+k & \equiv & 7\mod26
\end{eqnarray*}
(both mod 26). But we don't know yet how to solve systems of congruences!
\end{itemize}
\end{example}

\begin{xca}
If this were a system of \textbf{equations}
\begin{align*}
4m+k & =6\\
19m+k & =7
\end{align*}
how would you solve it for $m$ and $k$? 
\end{xca}

\begin{itemize}
\item \textbf{Goal}: figure out what it means to divide $\mod26$.
\item Well, dividing normally is the same as multiplying by the \emph{multiplicative
inverse} of a number. What's a multiplicative inverse?
\item For example, the multiplicative inverse of $3$ is $\frac{1}{3}$,
so dividing by $3$ is the same as multiplying by $\frac{1}{3}$.
And $3\times\frac{1}{3}=1$.
\end{itemize}
\begin{defn}
The \textbf{multiplicative inverse of $a\mod n$ }is the number $a^{-1}$
so that $aa^{-1}\equiv1\mod n$, if such an $a^{-1}$ exists.
\end{defn}

\begin{xca}
\textbf{(Reading Question) }The goal of this assignment is to make
yourself a \textquotedbl cheat sheet\textquotedbl{} which will tell
you the multiplicative inverse of any number mod 26, assuming it exists.
My expectation is not that you memorize such a list (though more power
to you if you'd like to); I'll give you enough time to re-derive each
multiplicative inverse using trial-and-error or the Extended Euclidean
Algorithm on a quiz or exam.
\end{xca}

\begin{enumerate}
\item Make a table as follows, with the numbers in the left column going
from 0 to 25. We'll fill in the blanks throughout this assignment.

\begin{tabular}{|c|c|c|}
\hline 
$a$ & Does $a^{-1}$ exist $\mod26$? & If so, what is $a^{-1}\mod26$?\tabularnewline
\hline 
\hline 
$0$ & N & \tabularnewline
\hline 
$1$ & Y & $1$\tabularnewline
\hline 
$2$ & N & \tabularnewline
\hline 
$3$ & Y & $9$\tabularnewline
\hline 
$4$ & N & \tabularnewline
\hline 
$5$ & Y & $21$\tabularnewline
\hline 
$6$ & N & \tabularnewline
\hline 
$7$ & Y & $15$\tabularnewline
\hline 
$8$ & N & \tabularnewline
\hline 
$9$ & Y & $3$\tabularnewline
\hline 
$10$ & N & \tabularnewline
\hline 
$11$ & Y & $19$\tabularnewline
\hline 
$12$ & N & \tabularnewline
\hline 
$13$ & N & \tabularnewline
\hline 
$14$ & N & \tabularnewline
\hline 
$15$ & Y & $7$\tabularnewline
\hline 
$16$ & N & \tabularnewline
\hline 
$17$ & Y & $23$\tabularnewline
\hline 
$18$ & N & \tabularnewline
\hline 
$19$ & Y & $11$\tabularnewline
\hline 
$20$ & N & \tabularnewline
\hline 
$21$ & Y & $5$\tabularnewline
\hline 
$22$ & N & \tabularnewline
\hline 
$23$ & Y & $17$\tabularnewline
\hline 
$24$ & N & \tabularnewline
\hline 
$25$ & Y & $25$\tabularnewline
\hline 
\end{tabular}
\end{enumerate}

\section{Decrypting affine ciphers with known key}
\begin{xca}
{[}slide with letter-to-number chart{]} Decrypt the ciphertext ``HXBOO
HAMOE PAPO'' given that it was encrypted using an affine cipher with
equation $C\equiv3P+2\mod26$. {[}Hint: subtract the $2$ from both
sides. How do we ``cancel the $3$'' given that we can't divide
by $3$? Division by $3$ is the same as multiplication by $3^{-1}\mod26$...{]}
\begin{itemize}
\item We start with the equation $C\equiv3P+2\mod26$ and we want to solve
it for $P$. 
\item First, we subtract $2$ from both sides to get $C-2\equiv3P\mod26$. 
\item Now, we want to ``divide by $3$'', which isn't allowed in modular
arithmetic. But this is the same as multiplying by $3^{-1}$, which
we saw in your Reading Question was $9$. So we multiply both sides
of the congruence by $9$ to get
\[
P\equiv9(3P)\equiv9(C-2)\mod26.
\]
\item To decrypt this ciphertext, we plug each letter in for $C$, then
record the values of $P$ we get, in order. The table from before
is useful:

\begin{tabular}{|c|c|c|c|c|c|c|c|c|c|c|c|c|c|c|}
\hline 
Ciphertext letter & H & X & B & O & O & H & A & M & O & E & P & A & P & O\tabularnewline
\hline 
\hline 
Ciphertext number $C$ & $7$ & $23$ & 1 & 14 & 14 & 7 & 0 & 12 & 14 & 4 & 15 & 0 & 15 & 14\tabularnewline
\hline 
Plaintext number $P\equiv9(C-2)$ & $45\equiv19$ & $9(21)\equiv9(-5)=-45\equiv-19\equiv7$ & $9(-1)\equiv17$ & ... &  &  &  &  &  &  &  &  &  & \tabularnewline
\hline 
Plaintext letter & t & h & r &  &  &  &  &  &  &  &  &  &  & \tabularnewline
\hline 
\end{tabular}
\item Solution: ``three times nine'' (which is $27\equiv1$, hence $9=3^{-1}\mod26$
\end{itemize}
\end{xca}

\begin{example}
What affine ciphers work for ASCII given that ASCII is $\mod256$?
\begin{itemize}
\item Anything with odd multiplicative key
\end{itemize}
\end{example}

\begin{itemize}
\item Even if we can tell which affine ciphers ``work'', we still don't
have a quick or even computable way to decrypt an affine cipher when
we know the key!
\end{itemize}
\begin{example}
In order to decrypt $C\equiv7P\mod29$ (using the cipher alphabet
with !?, appended), we need to multiply both sides by some number
$x$ so that $7x\equiv1\mod29$. How do we systematically find $x$?
\end{example}

\begin{itemize}
\item Next time, we'll learn how to use the Euclidean algorithm to decrypt
affine ciphers without having to guess and check!
\end{itemize}
\begin{xca}
{[}if time{]} Python programming
\end{xca}


\section{Python module help: BruteShift and AffineEncrypt}
\begin{xca}
We've defined three Python functions so far, and I'd like to take
the chance to look back on them and parse out what exactly each line
of code is doing. I know Python has been confusing, so I'm hoping
this will help.

In your groups, take the Python function you're assigned and make
two columns on a board:
\begin{enumerate}
\item one for the ``human steps'' for executing the function on example
inputs (e.g. plaintext and key for ShiftEncrypt, ciphertext and key
for ShiftDecrypt, ciphertext for BruteShift)
\item one for the ``human steps'' for executing the function on arbitrary
inputs (e.g. if someone gave you \emph{any} plaintext, what would
you do to encrypt it using \emph{any} key?)
\item one for each line of code under your ``def'' command for your function.
\end{enumerate}
Then try to match up each entry of column 3 to an entry of columns
1/2. 
\end{xca}

\begin{itemize}
\item \textbf{our library of Python functions:}
\begin{itemize}
\item \textbf{``}for \_\_\_ in \_\_\_\_'': repeat the code indented below
for every (first blank) in (second blank).
\begin{example}
for plaintext\_letter in plaintext: do the following indented code
for every letter in our plaintext.
\end{example}

\item (variable name) = \_\_\_\_\_: every time from now on that I type (variable
name), replace it with \_\_\_\_.
\begin{example}
alph = ``ABCD...YZ'': the variable ``alph'' means all capital
English letters from now on.
\end{example}

\item alph.find('letter'): find what position 'letter' is in the list ``alph''
and output that position.
\begin{example}
alph.find('F'): count up from $A=0$ in the list 'alph' (draw) until
we reach an F. Since F is in position $5$ (since we started our count
at $0$), output the number $5$.
\end{example}

\item Mod($a,n$): reduce $a\mod n$. (Also $a\%n$ does the same thing.)
\begin{example}
Mod($13,7)=6$. $7\%6=1$.
\end{example}

\item string += thing to add to string: literally think of taking a string,
poking a hole in thing, and stringing ``thing'' onto ``string''.
\begin{itemize}
\item plaintext += 'D': add a D to the end of our plaintext.
\end{itemize}
\item def FunctionName(input1, input2): whenever I write FunctionName and
some inputs, run everything tabbed over underneath this line on those
inputs.
\begin{itemize}
\item Ex.: def BruteShift(ciphertext): from now on, if I say BruteShift('DOFCOMWUYMUL'),
I want you to run all the code that brute-forces the ciphertext 'DOFCOMWUYMUL'.
\end{itemize}
\item (function name)(input1, input2): runs the function on these two inputs
\begin{example}
ShiftEncrypt('APPLE', 4) encrypts the word 'APPLE' using a shift of
$+4$.
\end{example}

\item string{[}number{]}: output whatever's in the position ``number''
in the string.
\begin{itemize}
\item alph{[}10{]}: find the letter corresponding to the number $10$ in
our letter-to-number chart.
\end{itemize}
\end{itemize}
\item more unsurprisingly:
\begin{itemize}
\item $a+b$: add the number $a$ to the number $b$
\item $a-b$: subtract $b$ from $a$.
\end{itemize}
\end{itemize}

\subsection{ShiftEncrypt}

\begin{tabular}{|c|c|}
\hline 
\textbf{Human} & \textbf{Python example (first with specific plaintext/key)}\tabularnewline
\hline 
\hline 
1. Input the plaintext. & \texttt{plaintext = ``MONKEY'', key = 4}\tabularnewline
\hline 
2. Input the cipher alphabet. & \texttt{alph = ``ABCDEFGHIJKLMNOPQRSTUVWXYZ''}\tabularnewline
\hline 
3. Convert the first plaintext letter to a plaintext number. & \texttt{plaintext\_number = alph.find('M')}\tabularnewline
\hline 
4. Add the key to the plaintext number, mod $26$. & \texttt{ciphertext\_number = Mod(plaintext\_number + key, 26)}\tabularnewline
\hline 
5. Convert the ciphertext number back to a letter. & \texttt{ciphertext\_letter = alph{[}ciphertext\_number{]}}\tabularnewline
\hline 
6. Append the letter to the ciphertext. & \texttt{ciphertext += ciphertext\_letter}\tabularnewline
\hline 
 & \tabularnewline
\hline 
 & \tabularnewline
\hline 
\end{tabular}

\subsection{ShiftDecrypt}

\begin{tabular}{|c|c|}
\hline 
\textbf{Human} & \textbf{Python example (first with specific ciphertext/key)}\tabularnewline
\hline 
\hline 
1. Input the ciphertext and key. & \texttt{ciphertext = ``DOFCOMWUYMUL'', key = 17}\tabularnewline
\hline 
2. Input the cipher alphabet. & \texttt{alph = ``ABCDEFGHIJKLMNOPQRSTUVWXYZ''}\tabularnewline
\hline 
3. Convert the first ciphertext letter to a number. & \texttt{ciphertext\_number = alph.find(ciphertext\_letter)}\tabularnewline
\hline 
4. Shift the ciphertext number back by the key, mod 26. & \texttt{plaintext\_number = Mod(ciphertext\_number - key, 26)}\tabularnewline
\hline 
5. Convert the plaintext number back to a letter. & \texttt{plaintext\_letter = alph{[}plaintext\_number{]}}\tabularnewline
\hline 
6. Append the plaintext letter to our plaintext. & \texttt{plaintext += plaintext\_letter}\tabularnewline
\hline 
 & \tabularnewline
\hline 
 & \tabularnewline
\hline 
\end{tabular}

\subsection{BruteShift}
\begin{xca}
List out steps, in English, that would allow you to decrypt a shift
cipher by testing every possible key.
\end{xca}

\begin{tabular}{|c|c|}
\hline 
\textbf{Human} & \textbf{Python example (first with specific ciphertext)}\tabularnewline
\hline 
\hline 
1. Input the ciphertext. & \texttt{ciphertext = ``DOFCOMWUYMUL''}\tabularnewline
\hline 
2. Input the cipher alphabet. & \texttt{alph = ``ABCDEFGHIJKLMNOPQRSTUVWXYZ''}\tabularnewline
\hline 
3. Convert the first ciphertext letter to a number. & done as part of ShiftDecrypt\tabularnewline
\hline 
4. Try decrypting the ciphertext using every shift. & \texttt{for key in range(0,26): }\tabularnewline
\hline 
 & \texttt{ShiftDecrypt(ciphertext, key)}\tabularnewline
\hline 
 & \tabularnewline
\hline 
 & \tabularnewline
\hline 
 & \tabularnewline
\hline 
\end{tabular}

\section{The connection between gcds and encryption}
\begin{itemize}
\item Why on earth does it matter whether the multiplicative key shares
a factor with the modulus when encrypting with an affine cipher?
\end{itemize}
\begin{example}
{[}slide{]} Look at the $\mod9$ table. Which numbers have multiplicative
inverses $\mod9$?
\begin{itemize}
\item $0^{-1}$? There's nothing that, when multiplied by $0$, gives $1$,
so $0$ doesn't have an inverse.
\item $1^{-1}$? $=1$
\item $2^{-1}=5$
\item $3^{-1}$ doesn't exist
\item $4^{-1}=7$
\item $5^{-1}=2$
\item $6^{-1}$ doesn't exist
\item $7^{-1}=4$
\item $8^{-1}=8$.
\end{itemize}
\end{example}

\begin{itemize}
\item What's so special about $3$ and $6$ here? The rows (or columns)
for 3 and 6 in the table contain only multiples of $3$. Why is that
the case?
\item We've seen that, if the multiplicative key shares any factors (other
than $1$) with the modulus, the affine cipher is no longer monoalphabetic. 
\end{itemize}
\begin{defn}
Recall that a number is \textbf{prime} if the only integers that divide
it are $1$ and itself. We'll use $p$ and $q$ to represent prime
numbers. \textbf{Composite numbers} are integers greater than one
that are not prime.
\end{defn}

\begin{itemize}
\item You'll remember that every positive whole number has a unique \textbf{prime
factorization}, i.e. a way to write it as a product of prime numbers.
This fact is called the \textbf{Fundamental Theorem of Arithmetic.}
\item Remember as well that we say an integer $d$ divides an integer $a$
if $a/d$ is a whole number, i.e. $d$ goes evenly into $a$ with
no remainder.
\item A \textbf{common divisor }of two integers $a$ and $b$ is an integer
(positive or negative) that divides both $a$ and $b$. 
\end{itemize}
\begin{defn}
If $a$ and $b$ are not both zero, then the \textbf{greatest common
divisor }of $a$ and $b$ is the largest positive integer that divides
both $a$ and $b$. We denote the greatest common divisor of $a$
and $b$ by $\gcd(a,b)$.
\end{defn}

\begin{xca}
Compute the following gcds:
\begin{enumerate}
\item $\gcd(13,26)$
\item $\gcd(55,165)$
\item $\gcd(253,598)$.
\begin{enumerate}
\item How did you do it? Most people factor both numbers: $253=11\times23$
and $598=2\times13\times23$, hence
\[
\gcd(253,598)=23.
\]
\item Factoring becomes annoying when we have big numbers. Fortunately,
later we'll see a better way to compute gcds and hence find out which
affine ciphers work.
\end{enumerate}
\end{enumerate}
\end{xca}

\begin{example}
For example, the cipher $C\equiv13P+1\mod26$ that we saw earlier
only has \textbf{two }possible ciphertext letters, $B$ and $O$.
\end{example}

\begin{itemize}
\item This is because we can't get every possible ciphertext letter out.\textbf{
In the example above, we can only get out the ciphertext letters that
are multiples of $13$.}
\item This is true in general: anytime the multiplicative key and the modulus
share any factors, the only ciphertext letters that come out will
be multiples of those factors.
\item \textbf{This isn't enough ciphertext letters to be a good cipher},
because it ``collapses'' multiple plaintext letters into a single
ciphertext letter.
\end{itemize}
\begin{thm}
If $a$ is a whole number between $0$ and $n-1$, then $\gcd(a,n)$
goes evenly into any number that's congruent to $a\mod n$.
\end{thm}

\begin{proof}
Assume some number $x$ is congruent to $4\mod12$. Then
\[
x\div12=yR4
\]
where $y$ is some whole number. So $x$ is some multiple of $12$
plus $4$; we can write $x=12y+4$, where $y$ is some whole number.
Now, $\gcd(4,12)=4$ and $x=4(3y+1)$ is divisible by $4$. 

In general, if $x\equiv a\mod n$, then $x=ny+a$, where $y$ is some
whole number. Then $\gcd(a,n)$ is in particular a divisor of both
$a$ and $n$, meaning that $(ny+a)/\gcd(a,n)$ is a whole number.
This means that $x/\gcd(a,n)$ is a whole number, hence $\gcd(a,n)\mid x$.
\end{proof}
\begin{cor}
If $\gcd(a,n)\neq1$, then no multiple of $a$ is congruent to $1\mod n$.
In other words, we can't get every number in row $a$ of the $\mod n$
multiplication table, and \textbf{$a$ does not have a multiplicative
inverse $\mod n$.}
\end{cor}

\begin{proof}
Suppose that some multiple of $a$, say $ka$, is congruent to $1\mod n$.
By the Theorem, this implies that $\gcd(a,n)\mid1$, and since $\gcd(a,n)\geq1$,
this shows that we must have $\gcd(a,n)=1$. By this logic, if $\gcd(a,n)\neq1$,
we couldn't have gotten $1\mod n$ as a multiple of $a$, so $a$
has no multipicative inverse $\mod n$, and in particular $1$ doesn't
show up in $a$'s row of the $\mod n$ multiplication table.
\end{proof}
\begin{cor}
The following are equivalent for any whole numbers $m,n$:
\begin{enumerate}
\item $\gcd(m,n)=1$.
\item any affine cipher with multiplicative key $m\mod n$ is decryptable
by the intended recipient.
\item every number between $0$ and $n-1$ is a multiple of $m\mod n$.
\item $m$ has a multiplicative inverse $m^{-1}\mod n$.
\end{enumerate}
\end{cor}

\begin{example}
Consider the affine cipher $C\equiv13P\mod26$. Note that $\gcd(13,26)=13$.
Any plaintext letter that we plug in will spit out some multiple of
$13$. Reduced mod $26$, these multiples of $13$ all become either
$13$ or $0$. And $13=\gcd(13,26)$ divides both $13$ and $0$ evenly.
Since no multiple of $13$ can be congruent to $1\mod26$, $13$ does
not have a multiplicative inverse $\mod26$.
\end{example}

\begin{itemize}
\item Though sometimes we need slightly fancy math machinery to break affine
ciphers, sometimes we can do so by trial and error.
\end{itemize}

\section{Finding gcds quickly using the Euclidean algorithm}
\begin{defn}
Recall that we write $a\mid b$ if $a$ goes evenly into $b$ with
no remainder.

If $a$ and $b$ are not zero, then the \textbf{greatest common divisor}
of $a$ and $b$, called $\gcd(a,b)$, is the largest positive integer
that divides both $a$ and $b$.

If $\gcd(a,b)=1$, we say that $a$ and $b$ are \textbf{relatively
prime}.
\end{defn}

\begin{example}
How to find the gcd of 261 and 231 quickly:
\begin{enumerate}
\item Notice that if a number goes evenly into both $261$ and $231$, then
it goes evenly into $261-231$ as well. (For example, since $\frac{261}{3}=87$
and $\frac{231}{3}=77$, 
\[
\frac{261-231}{3}=\frac{261}{3}-\frac{231}{3}=87-77=10R0
\]
and since $10$ is a whole number, $3\mid(261-231)$.) This means
that the biggest number that goes into both $261$ and $231$ must
also go into $30$. So $\gcd(261,231)=\gcd(30,231)$.
\item Repeat this process for $231$ and $30$. Any number that goes evenly
into both $231$ and $30$ must also go evenly into $231-30$. Hence
$\gcd(30,231)=\gcd(30,201)$. But this doesn't help all that much!
We can keep subtracting $30$ from $231$ over and over until we get
a number smaller than $30$:
\[
\gcd(30,231)=\gcd(30,201)=\gcd(30,171)=\gcd(30,141)=\dots
\]
but it's a lot faster to just figure out how many times $30$ goes
into $231$ and take the remainder, since that's the same number that
will be left over after we subtract all the $30$s out of $231.$
In other words, notice that 
\[
231=30(7)+21
\]
so that if we subtract all the $30$s out of $231$, we end up subtracting
seven $30$s, and we're left with $21$. Hence
\[
\gcd(30,231)=\gcd(30,21).
\]
\item Now, $\gcd(30,21)$ is a lot easier to compute than $\gcd(261,231)$!
You can probably just read off that $\gcd(30,21)=3$. But say you
didn't remember that offhand, or you ended up with a gcd that is harder
to compute. Let me keep going to make a point:
\begin{align*}
30 & =21(1)+9\text{ so }\gcd(30,21)=\gcd(9,21)\\
21 & =9(2)+3\text{ so }\gcd(9,21)=\gcd(9,3)\\
9 & =3(3)+0\text{ so }\gcd(9,3)=\gcd(0,3).
\end{align*}
\end{enumerate}
\end{example}

\begin{xca}
This is a weird step, right? What should the gcd of $0$ and any other
number be?
\begin{itemize}
\item Since any number goes evenly into $0$\textemdash for example, $\frac{0}{3123}=0$
so $3123\mid0$\textemdash $\gcd(0,3)=3$. 
\item In general, if $x$ is any whole number, $\gcd(x,0)=x$.
\item The method above will always end with a remainder of zero. Then the
$\gcd$ we're looking for is just the other number.
\end{itemize}
\end{xca}

\begin{defn}
The process above is called the \textbf{Euclidean algorithm}. It was
described by the mathematician Euclid, who basically invented geometry
around $300$ B.C.
\end{defn}

\begin{itemize}
\item Why are gcd's important? Well, we want to easily be able to tell which
affine ciphers work, at the very least.
\end{itemize}
\begin{xca}
\textbf{{[}RQ{]} }Use the Euclidean Algorithm to determine which of
the following affine ciphers works to create decryptable ciphertext.
Explain your answers.
\begin{enumerate}
\item (Khmer/Cambodian alphabet) $C\equiv17P+4\mod74$
\begin{enumerate}
\item Works:
\begin{align*}
74 & =17(4)+6\\
17 & =6(2)+5\\
6 & =5(1)+1\\
5 & =1(5)+0.
\end{align*}
The $\gcd$ is just the last remainder before the $0$, so $\gcd(17,74)=1$
and this cipher works for the Cambodian alphabet!
\end{enumerate}
\item $C\equiv365P\mod755$
\begin{enumerate}
\item Doesn't work, $365$ and $755$ share a factor of $5$
\end{enumerate}
\item $C\equiv87P\mod1364$
\begin{enumerate}
\item $1364=87(15)+69$
\item $87=69(1)+18$
\item $69=18(3)+15$
\item $18=15(1)+3$
\item $15=3(5)$ so $\gcd(87,1364)=3$ and the cipher doesn't work
\end{enumerate}
\item Unicode is an updated version of ASCII. Unicode gives a way of encoding
information on a computer with 1,114,112 possible numbers. Only about
$10\%$ of them are currently in use; say there are $111,411$ Unicode
numbers in use. Does the affine encryption
\[
C\equiv103,011P+34,423\mod111,411
\]
work to encrypt Unicode characters?
\begin{enumerate}
\item $111411=103011(1)+8400$
\item $103011=8400(12)+2211$
\item $8400=2211(3)+1767$
\item $2211=1767(1)+444$
\item $1767=444(3)+435$
\item $444=435(1)+9$
\item $435=9(48)+3$
\item $9=3(3)+0$
\item So $\gcd(103011,111411)=3\neq1$. Hence this cipher doesn't work.
\end{enumerate}
\end{enumerate}
\end{xca}


\section{Python work time: AffineEncrypt}
\begin{itemize}
\item Walk through changes from ShiftEncrypt code
\item Emphasize need for third input (now there are two keys! Each key is
a separate input.)
\end{itemize}

\end{document}
