%% LyX 2.3.6.2 created this file.  For more info, see http://www.lyx.org/.
%% Do not edit unless you really know what you are doing.
\documentclass[oneside,english]{amsart}
\usepackage[T1]{fontenc}
\usepackage{geometry}
\geometry{verbose,tmargin=1in,bmargin=1in,lmargin=1in,rmargin=1in}
\setcounter{secnumdepth}{2}
\setcounter{tocdepth}{2}
\usepackage{amstext}
\usepackage{amsthm}
\usepackage{graphicx}

\makeatletter

%%%%%%%%%%%%%%%%%%%%%%%%%%%%%% LyX specific LaTeX commands.
%% Because html converters don't know tabularnewline
\providecommand{\tabularnewline}{\\}

%%%%%%%%%%%%%%%%%%%%%%%%%%%%%% Textclass specific LaTeX commands.
\numberwithin{equation}{section}
\numberwithin{figure}{section}
\theoremstyle{plain}
\newtheorem{thm}{\protect\theoremname}
\theoremstyle{definition}
\newtheorem{example}[thm]{\protect\examplename}
\theoremstyle{definition}
\newtheorem{xca}[thm]{\protect\exercisename}
\theoremstyle{plain}
\newtheorem{prop}[thm]{\protect\propositionname}

\makeatother

\usepackage{babel}
\providecommand{\examplename}{Example}
\providecommand{\exercisename}{Exercise}
\providecommand{\propositionname}{Proposition}
\providecommand{\theoremname}{Theorem}

\begin{document}
\title{Breaking Affine Ciphers without the Key}
\maketitle

\section{The Euclidean Algorithm in Python}
\begin{example}
The Euclidean Algorithm. Let's follow the algorithm to compute $\gcd(2,5)$
and figure out what we want to program in Python. I'll do things slightly
differently to make it programmable, but note that the algorithm is
still basically the same. 

Here $a=2$ and $b=5$.

\begin{tabular}{|c|c|}
\hline 
to compute $\gcd(2,5)$ & to compute $\gcd(a,b)$ if $a<b$\tabularnewline
\hline 
\hline 
Reduce $5\mod2$ to get $1$. & Reduce $b\mod a$.\tabularnewline
\hline 
Replace $b=5$ with $b=1$. & Replace $b$ with $b\mod a$.\tabularnewline
\hline 
Switch $a=2$ and $b=1$ so that $a=1<2=b$. & Switch $a$ and the new $b$ so that $a<b$.\tabularnewline
\hline 
Reduce $2$ mod $1$ to get $0$. & Reduce the new $b$ mod the new $a$.\tabularnewline
\hline 
Replace $b=2$ with $b=0$. & Replace $a$ with $a\mod b$.\tabularnewline
\hline 
Switch $a=1$ and $b=0$ so that $a=0<1=b$. & Switch $a$ and the new $b$ so that $a<b$.\tabularnewline
\hline 
Stop the algorithm now that $a=0$. & ...stop the algorithm when the smaller number $a=0$.\tabularnewline
\hline 
The gcd is the last nonzero remainder, $b=1$. & The gcd is the last nonzero remainder, $b$.\tabularnewline
\hline 
\end{tabular}.

\begin{tabular}{|c|c|}
\hline 
to compute $\gcd(a,b)$ if $a<b$ & \textbf{code {[}ask the class{]}}\tabularnewline
\hline 
\hline 
Reduce $b\mod a$. & $b\%a$\tabularnewline
\hline 
Replace $b$ with $b\mod a$. & $b=b\%a$\tabularnewline
\hline 
Switch $a$ and the new $b$ so that $a<b$. & $a,b=b,a$\tabularnewline
\hline 
Reduce the new $b$ mod the new $a$. & $b\%a$\tabularnewline
\hline 
Replace $b$ with $b\mod a$. & $b=b\%a$\tabularnewline
\hline 
Switch $a$ and the new $b$ so that $a<b$. & $a,b=b,a$\tabularnewline
\hline 
...stop the algorithm when the smaller number $a=0$. & Put this repeated process under ``while $a$ != $0$:''\tabularnewline
\hline 
The gcd is the last nonzero remainder, $b$. & at the end of the while loop: ``return b''\tabularnewline
\hline 
\end{tabular}

\end{example}

\begin{xca}
Finish defining the $\gcd$ function and using it to compute the gcds
in CoCalc.
\end{xca}

\textbf{Last Time}: we've been working on the following process for
solving affine ciphers if we knew the key (e.g. we were the intended
recipient):
\begin{enumerate}
\item Define ``division'' in modular arithmetic, where we can.
\item Learn when multiplicative inverses exist modulo $n$.
\item Learn how to find the multiplicative inverse of a number modulo $n$.
(Recall this requires the Extended Euclidean Algorithm, as below.)
\begin{enumerate}
\item Do the regular Euclidean algorithm on $a$ and $b$ to get $\gcd(a,b)$.
Write all your steps on separate rows on the left side of the page.
\item On the right side of the page, solve each equation for the remainder
at each step.
\item Begin with the bottom equation and substitute the remainders in ascending
order. Simplify at each step, being careful not to explicitly multiply
the remainders or the original two numbers.
\end{enumerate}
\end{enumerate}
\begin{itemize}
\item But what if we intercept a message from a hostile foreign government
that we know was encrypted with an affine cipher. Naturally, they
wouldn't just hand us the keys. What do we do?
\item We already know how to use frequency analysis to guess which letter
corresponds to plaintext ``e'', ``t'', ``a'', or another common
letter.
\item Now let's bring everything together to break affine ciphers with unknown
key!
\end{itemize}

\section{Solving linear congruences}

We know how to solve equations like $C\equiv15P\mod26$ for $P$ by
multiplying both sides by the multiplicative inverse of $15\mod26$,
which is $7$:
\[
7C\equiv(7\cdot15)P\equiv P\mod26.
\]

\begin{example}
But now let's say that we're trying to crack the affine cipher. 
\end{example}

\begin{xca}
Encrypt the message ``etta eats eels'' using the affine cipher with
$m=-1$ and $k=-1$. 
\end{xca}

\begin{example}
Let's take the resulting ciphertext, VGGZ VZGH VVOH, assume we don't
know the keys, and try to solve for $m$ and $k$.
\begin{itemize}
\item The most common letters are $V$ and $G$, which appear seven and
four times, respectively. This suggests that the ciphertext V and
G correspond to plaintext e and t, respectively. Let's write the previous
statement in number form:
\item Ciphertext 21 corresponds to plaintext 4, and ciphertext 6 corresponds
to plaintext 19.
\item Let's look at the equation $(\ast)$ above. When $C=21$, then $P=4$
and the equation becomes
\[
21\equiv4m+k.
\]
When $C=6$, then $P=19$ and the equation becomes
\[
6\equiv19m+k.
\]
\item Together, these equations give us a \textbf{system of congruences}
\begin{eqnarray*}
4m+k & \equiv & 21\mod26\\
19m+k & \equiv & 6\mod26.
\end{eqnarray*}
But we don't know yet how to solve systems of congruences!
\end{itemize}
\end{example}

\begin{xca}
\textbf{(reading question) }If this were a system of \emph{equations}
like the ones you saw in Algebra II, how would you solve the system?
In other words, solve the system of equations
\begin{align*}
4m+k & =21\\
19m+k & =6.
\end{align*}
\begin{itemize}
\item We subtract the second equation from the first to get
\[
-15m\equiv15\mod26
\]
\item We notice right away that $m\equiv-1$ is the solution to this equation!
\item It's not always this easy, though. What if we'd guessed instead that
$e\mapsto G$ and $a\mapsto V$? Then our system becomes
\begin{align*}
6 & \equiv4m+k\mod26\\
21 & \equiv0m+k\mod26
\end{align*}
so that right away we say that $k=21$. Subtracting $k=21$ from both
sides of the top equation gives
\[
4m\equiv-15\equiv11\mod26
\]
which actually has no solution!
\end{itemize}
\end{xca}

\begin{itemize}
\item \textbf{Ground rule}: stick to guessing what $e$ and $t$ go to when
attempting to crack affine ciphers and you'll be safe.
\end{itemize}
\begin{xca}
Solve the following systems of congruences $\mod26$ for $m$ and
$k$:
\begin{enumerate}
\item 
\begin{align*}
3 & \equiv13m+k\\
2 & \equiv12m+k
\end{align*}

Subtracting equation $2$ from equation $1$ gives $1\equiv m$ immediately.
Plugging in $m=1$ into the first equation gives $3\equiv13+k\mod26$,
hence $k=-10\equiv16\mod26$. 
\item 
\begin{align*}
17 & \equiv23m+k\\
12 & \equiv16m+k
\end{align*}

Subtracting the second equation from the first gives $5\equiv7m$.
Now we want to solve for $m$, which means we need to multiply both
sides by $7^{-1}=15$:
\[
15\times5=15\times7m\equiv m\mod26
\]
Since $15\equiv-11\mod26$, $15\times5\equiv-11\times5=-55\equiv-3\mod26$.
So $m=-3\equiv23\mod26$. We plug in $m=-3$ to the second equation
to solve for $k$:
\begin{align*}
12 & \equiv16(-3)+k\mod26\\
12 & \equiv-48+k\mod26\\
60 & \equiv k\mod26\\
8 & \equiv k\mod26.
\end{align*}

\item 
\begin{align*}
25 & \equiv15m+k\\
12 & \equiv2m+k
\end{align*}

\begin{itemize}
\item Note: this one simplifies to $13\equiv13m\mod26$. This could have
the solution $m\equiv1\mod26$, of course, but it also has the solutions
\[
m\equiv3,5,7,9,11,13,15,17,19,21,23,25\mod26
\]
\end{itemize}
\end{enumerate}
\end{xca}

\begin{prop}
\textbf{We can't solve systems of congruences mod 26 if we end up
with a congruence $am\equiv b\mod26$ where $\gcd(a,26)\neq1$! }The
problem is that $a^{-1}$ doesn't exist $\mod26$, and so we can end
up with multiple possible answers for $m$.
\end{prop}

\begin{xca}
{[}slide{]} Decipher the affine ciphertext \textbf{hikctyhghykaezztgaayrhbggrvgdyfg
}given that the most common ciphertext letters are G and H, respectively.
\begin{itemize}
\item The system is 
\begin{align*}
6 & \equiv4m+k\mod26\\
7 & \equiv19m+k\mod26
\end{align*}
\item Subtracting the first equation from the second gives
\[
1\equiv15m\mod26
\]
\item Multiplying by $15^{-1}=7\mod26$ on both sides yields
\[
7\equiv(7\times15)m\equiv m\mod26.
\]
\item Plugging in $m=7$ to the first equation yields
\begin{align*}
6 & \equiv4(7)+k\mod26\\
 & \equiv2+k\mod26\\
4 & \equiv k\mod26
\end{align*}
so that our encryption equation was
\[
C\equiv7P+4\mod26.
\]
\item This means our decryption equation is
\begin{align*}
C-4 & \equiv7P\mod26\\
15(C-4) & \equiv(15\cdot7)P\equiv P\mod26\\
15C-60\equiv15C-8 & \equiv P\mod26.
\end{align*}
\item Plugging in yields the plaintext
\[
\text{tim wrote toms address on the envelope}.
\]
\end{itemize}
\end{xca}

\textbf{Solving Affine Ciphers}:
\begin{enumerate}
\item Form a system of linear congruences to solve for $m$, the multiplicative
constant, and $k$, the additive constant.
\item Simplify the system of congruences by substitution, elimination, or
addition/subtraction of congruences.
\item Find the solution of the resulting congruence using trial and error
or the extended Euclidean algorithm. 
\item Use $m$ and $k$ to decrypt the ciphertext using the formula $p_{i}=m^{-1}(c_{i}-k)$
mod $26$.
\end{enumerate}

\subsection{Python Module: Frequency Analysis}
\begin{itemize}
\item In order to break an affine (or any monoalphabetic) cipher, we have
to do a frequency count.
\item For longer messages, it's very impractical to frequency analyze by
hand. For example, consider what would happen if you intercepted the
entire Declaration of Independence enciphered with $C\equiv25P+5\mod26$.
\item Let's start to build a tool to frequency analyze in Python!
\item The algorithm we'll use is as follows: {[}show what would happen in
the message ``TEST''{]}
\begin{itemize}
\item \textbf{Goal}: build a list of ciphertext letters together with their
corresponding counts in the message.
\begin{enumerate}
\item Start with an empty count\_list.
\item Add 26 zeroes to the empty count\_list, one to represent each ciphertext
letter.
\item Go through the intercepted ciphertext message letter by letter and,
for each ciphertext letter in the message, add $1$ to that letter's
entry in count\_list.
\item Label each entry of count\_list with the corresponding ciphertext
letter.
\item Return the labeled count\_list to the user.
\end{enumerate}
\end{itemize}
\end{itemize}
\begin{xca}
{[}CoCalc{]} open the Frequency Analysis module and work through it,
step-by-step, with the class, giving them occasional tasks.
\end{xca}

\begin{itemize}
\item \textbf{Problem}: what if you have no idea what ``E'' is in your
message? 
\end{itemize}
\begin{example}
For instance, consider the message ``YZU RO YBOYZEBA YZUIZOND'',
which you're told is encrypted with an affine cipher. Just looking
at this message, your first guess would be that $e\mapsto Y$ or that
$e\mapsto Z$, both of which would be incorrect! Aside from attempting
a brute-force attack, it's very unclear how to decipher this message
without more information. Short messages, in general, are much harder
to decipher using frequency analysis.
\end{example}

\begin{xca}
Suppose you're told that the most common letters in the plaintext
message are z and i. Decrypt the message.
\begin{itemize}
\item zip on zanzibar ziplines
\end{itemize}
\textbf{{[}reading question{]} }See if you can decrypt your ``weird''
affine-enciphered messages if the author tells you the two most common
plaintext letters in each message. How easy or difficult is the process? 

{[}slide{]} Decrypt GYOMXNOGNG QUGN ETNMX MPLMZOMXYM K TMMJOXA XEN
TKZ ZMQEBMF TZEQ KJKZQ EX YEXNMQLJKNOXA NHM TJEEF ET XMI CXEIJMFAM
IHOYH MKYH WMKZ RZOXAG IONH ON given the frequency distribution: \includegraphics[scale=0.5]{\string"Old Cryptography Notes/pasted101\string".png}

SCIENTISTS MUST OFTEN EXPERIENCE A FEELING NOT FAR REMOVED FROM ALARM
ON CONTEMPLATING THE FLOOD OF NEW KNOWLEDGE WHICH EACH YEAR BRINGS
WITH IT

$C\equiv7P+10\mod26$

For the following, you can type the messages into an online frequency
analysis tool such as the one at dcode.fr/frequency-analysis to get
a frequency count. (Next week, you'll develop a frequency analysis
tool of your own in Python.) Then decipher each message.
\end{xca}

~
\begin{xca}
\textbf{{[}slide{]} }

\includegraphics[scale=0.5]{\string"Old Cryptography Notes/pasted102\string".png}
\end{xca}

~
\begin{xca}
\includegraphics[scale=0.5]{\string"Old Cryptography Notes/pasted73\string".png}

\includegraphics{\string"Old Cryptography Notes/pasted132\string".png}
\end{xca}


\section*{Read ``Cracking the Enigma'' in The Code Book}

\section*{Enigma Machine Videos: Numberphile}

https://www.youtube.com/watch?v=G2\_Q9FoD-oQ\&src\_vid=V4V2bpZlqx8\&feature=iv\&annotation\_id=annotation\_309534

\section*{Sample Enigma Encoding}

http://startpad.googlecode.com/hg/labs/js/enigma/enigma-sim.html

\section*{Evening Session}

\textbf{Journal entry. }Do you believe that cryptography and cryptographical
techniques can be primarily used for ``good'' or for ``evil''?
Explain.

Breaking Master Locks (see PDF)
\end{document}
