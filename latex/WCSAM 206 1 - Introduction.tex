%% LyX 2.3.6.2 created this file.  For more info, see http://www.lyx.org/.
%% Do not edit unless you really know what you are doing.
\documentclass[oneside,english]{amsart}
\usepackage[T1]{fontenc}
\usepackage{geometry}
\geometry{verbose,tmargin=1in,bmargin=1in,lmargin=1in,rmargin=1in}
\usepackage{amsthm}
\usepackage{graphicx}

\makeatletter
%%%%%%%%%%%%%%%%%%%%%%%%%%%%%% Textclass specific LaTeX commands.
\numberwithin{equation}{section}
\numberwithin{figure}{section}
\theoremstyle{plain}
\newtheorem{thm}{\protect\theoremname}
\theoremstyle{remark}
\newtheorem{claim}[thm]{\protect\claimname}

\makeatother

\usepackage{babel}
\providecommand{\claimname}{Claim}
\providecommand{\theoremname}{Theorem}

\begin{document}
\title{WCSAM 206 Day 1 - Introduction}
\author{Kenan Ince}

\maketitle


\section{Introduction}
\begin{itemize}
\item Name/pronouns/PhD from Rice, research in how hard it is to untie mathematical
knots

\begin{itemize}
\item Please feel free to talk to me about my research at any time during
the semester. 
\end{itemize}
\item My primary goal in teaching math is to communicate that math is a
creative, logical field with immense importance to society. A lot
of what we're teaching is logic, which is like a prosthetic you attach
to your reasoning to make it more powerful.
\end{itemize}

\subsubsection{Office and office hours}
\begin{itemize}
\item My office is Foster 311 (go up the stairs to the third floor, turn
right; I'm at the end of the hall.)
\item My office hours this semester will be MW 1-2pm; TTH 11am-12pm. Anyone
who can't make those hours? I'll usually be around the office 1-4pm
MW, 10am-1pm TTh, and (maybe) 11-noon Friday as well. Make appointments
at least $24$ hours in advance!
\item Tentative course outline: linked in Canvas under Syllabus
\end{itemize}

\subsubsection{Canvas}
\begin{itemize}
\item How to access Canvas
\begin{itemize}
\item When you log on, you're faced with the course modules
\item You must complete the previous module before starting the current
one
\item All course assignments will be posted in Canvas
\item You can look at your assignment grades by clicking Grades in the left-hand
menu
\end{itemize}
\item Sign up now for Canvas notifications (should be on by default; under
Account -> Notifications)
\begin{itemize}
\item Keep Due Date, Course Content, Announcement settings on ``notify
me right away'' (the checkmark button)
\end{itemize}
\item I can't guarantee I'll always remember to announce assignments in-class;
you're expected to check Canvas before and after every class period
to determine what assignments you have.
\item I will post all of your grades in Canvas.
\item We take the Title IX and disability provisions extremely seriously!
If you require any academic accommodations related to a disability,
please contact Karen Hicks ASAP.
\end{itemize}

\section{{[}slide{]} Introductory codebreaking}
\begin{itemize}
\item OK, let's get to some codebreaking! You can work in groups for these.
\end{itemize}
\begin{enumerate}
\item The following message has been encrypted with a ``shift cipher'',
i.e. by shifting each letter forward by some number of letters (and
once a letter hits Z, it loops back around to A). Decipher the message.

BJQHTRJYTBJXYRNSXYJWNMTUJDTZMFAJKZSNSYMNXHQFXX
\begin{itemize}
\item possible techniques: just trying a bunch of shifts, giving up, discovering
frequency analysis
\item {[}answer{]}: WELCOMETOWESTMINSTERIHOPEYOUHAVEFUNINTHISCLASS
\item tactics: most frequent letter = E
\end{itemize}
\item Try to decipher the following secret message.
\begin{itemize}
\item ykt vk jks boue sk le akkv/ykt vk jks boue sk wogf kj yktp fjeeq/rkp
o btjvpev hcgeq sbpktab sbe veqeps pemejscja/ykt kjgy boue sk ges
sbe qkrs ojchog kr yktp lkvyg/kue wbos cs gkueq
\begin{itemize}
\item possible techniques: trying a bunch of shifts, word spacing \& guessing
3-letter words, filling in the blanks
\item Keyword cipher with key ``OLIVER''
\item It's the following excerpt from the Mary Oliver poem ``Wild Geese'':\\
You do not have to be good. You do not have to walk on your knees
For a hundred miles through the desert, repenting. You only have to
let the soft animal of your body love what it loves.
\item tactic: same idea, not a shift so have to figure out each letter individually
\begin{itemize}
\item look at how much spacing helps!
\end{itemize}
\end{itemize}
\end{itemize}
\item Try to decipher the following message.

\includegraphics{pasted1}
\begin{itemize}
\item Hint 1: The ciphertext letters NYPVTT decrypt to BERLIN
\item Hint 2: The word immediately following BERLIN is CLOCK
\item Process: notice it's not monoalphabetic since $i,n\mapsto T$; get
frustrated, wonder about spacing...?
\end{itemize}
{[}this is Kryptos part IV; solve it and you'll be famous!{]}
\end{enumerate}
\begin{itemize}
\item Why do you think I gave you an unsolved code?
\begin{itemize}
\item What do you think mathematicians spend most of their time doing?
\item How do you define ``success'' when working on an unsolved problem?
\begin{itemize}
\item Making sense of a problem
\item Increasing depth of understanding (what methods DON'T work?)
\item How would you describe your process of engagement on this problem?
\end{itemize}
\item Failure is completely normal! 
\begin{itemize}
\item Even on problems that are solved, it often took years or even millennia
to solve them!
\item So don't feel ashamed or worried if you don't get something immediately.
\end{itemize}
\end{itemize}
\end{itemize}

\section{Setting the Stage}

Get in groups of size 3\textendash 4. Group members should introduce
themselves - name, pronouns, hobbies outside school, goals for the
semester/year, reason they're taking the course. For each of the questions
that follow, I will ask you to: 
\begin{enumerate}
\item Think about a possible answer on your own. 
\item Discuss your answers with the rest of your group. 
\item Share a summary of each group\textquoteright s discussion. 
\end{enumerate}

\subsubsection{Questions}
\begin{enumerate}
\item What are the goals of a liberal arts education? 
\item How does a person learn something new? 
\item What is the value of making mistakes in the learning process? 
\item How do we create a safe environment where risk taking is encouraged
and productive failure is valued? 
\end{enumerate}
\begin{itemize}
\item I will be teaching this course using active learning methods for all
the reasons you gave, as well as because research shows IBL improves
student understanding of mathematics more than traditional lecture-based
teaching methods.
\end{itemize}
\begin{quotation}
\textquotedblleft Any creative endeavor is built on the ash heap of
failure.\textquotedblright \textemdash Mike Starbird
\end{quotation}
\begin{claim}
An education must prepare a student to ask and explore questions in
contexts that do not yet exist. That is, we need individuals capable
of tackling problems they have never encountered and to ask questions
no one has yet thought of. 

If we really want students to be independent, inquisitive, \& persistent,
then we need to provide them with the means to acquire these skills.
\end{claim}

\begin{itemize}
\item Group roles: In order to facilitate group interaction, you will take
on group roles designed to mimic the role of mathematical and/or cryptographical
researchers. The roles are as follows \textbf{(follow along in your
syllabus)}:
\begin{itemize}
\item \textbf{The facilitator} is responsible for making sure every student
is able to contribute and be heard. \emph{Contributions} may include
asking good questions, rephrasing someone else's idea, coming up with
a way of connecting mathematics to the real world, and many others. 
\item the \textbf{resource manager} is responsible for obtaining and keeping
track of all necessary resources to solve a problem. \emph{Resources}
may include writing utensils, paper, the Internet, your instructor,
data sources, and most importantly, your team. 
\item the \textbf{lead author} is responsible for writing down the ideas
that each group comes up with. 
\item the \textbf{communicator} is responsible for reporting what your group
came up with to the class, instructor, and any relevant community
groups. 
\end{itemize}
\item Assign roles to each group by first letters of first name; they'll
rotate every day of class.
\item Anyone can answer questions posed to the whole class, but if your
group came up with an idea that hasn't been shared yet, it \textbf{someone}
in your group's responsibility to share your answer so that everyone
can learn from you!
\end{itemize}

\subsection{Engagement Points (Extra Credit Opportunities)}
\begin{itemize}
\item The Westminster College Department of Mathematics values the experiences
that students have with mathematics and data science outside of the
formal classroom. In order to encourage our students to find new and
interesting ways to engage with our disciplines, we have incorporated
a department-wide extra credit policy in which extracurricular activities
gain you \textquotedbl engagement points\textquotedbl{} which will
contribute toward your course grades..
\item Examples of activities that are likely to qualify include, but are
not limited to: attending office hours three times; attending a meeting
of the S-Cubed seminar; volunteering in the East High or Cottonwood
Tutoring Programs; attending Lemma social activities or meetings;
attending college Diversity, Equity, and Inclusion programming; taking
the Putnam exam; attending a screening of a film with mathematical
themes; participating in a Research Experience for Undergraduates
(REU) during the summer; attending a non-required math-related talk
or seminar; attending a regional/national math conference; giving
a talk related to math; submitting or publishing a paper in an academic
journal; participating in a math competition; reading a book with
mathematical themes; and many others.
\item If you participated in an activity that you think qualifies for engagement
points, please fill out the Google Form located at tinyurl.com/WMengagement.
Because we are unable to constantly check the form responses, please
contact me to determine whether you were granted engagement points
for your activity.
\item In this course, you may use three engagement points at any time to
get a free Pass on a reading response or five engagement points to
get a free Pass on a news response. I reserve the right to limit how
many times you may redeem engagement points in these ways. Please
send me an email or tell me in person if you'd like to use your engagement
points.
\end{itemize}

\subsubsection{Tutoring Programs}
\begin{itemize}
\item You may earn engagement points by volunteering your time in Westminster's
Cottonwood High or East High Tutoring Programs. This involves volunteering
at least three times to work at an after-school-tutoring lab at one
of the two high schools with which Westminster has a partnership.
There, you would help high school students struggling with mathematics.
The high-school students receiving this tutoring are usually in Algebra
I, Geometry, or Algebra II and see mathematics as confusing and frustrating.
Both Programs are looking for a willingness to help remove this fear
from a scary subject in their tutors.
\item Students who tutor or TA at least three times in the Cottonwood tutoring
program will receive a stipend of $\$8$ for every hour spent tutoring,
as well as one engagement point per hour volunteered.\} Please see
tinyurl.com/sherlockclub for further details and to sign up for Cottonwood
tutoring or teaching assistantship. 
\item We take the Title IX and disability provisions extremely seriously!
If you require any academic accommodations related to a disability,
please contact Karen Hicks ASAP.
\end{itemize}

\subsection{Group work}
\begin{itemize}
\item What rules should we set for groups?
\begin{itemize}
\item Whose voices are valued?
\item What philosophy do we have about who can make contributions to mathematics
and the value of those contributions?
\item How does this connect to the values we espouse for our democracy?
\end{itemize}
\item Anyone can answer questions posed to the whole class, but if your
group came up with an idea that hasn't been shared yet, it \textbf{someone}
in your group's responsibility to share your answer so that everyone
can learn from you!
\end{itemize}

\subsubsection{Growth mindset}
\begin{itemize}
\item \textquotedblleft How many of you feel, deep down in your most private
thoughts, that you aren\textquoteright t actually any good at math?
That by some miracle, you\textquoteright ve managed to fake your way
to this point, but you\textquoteright re always at least a little
worried that your secret will be revealed? That you\textquoteright ll
be found out?\textquotedblright{}
\item \textquotedblleft I want you to discuss the following question with
your groups\textendash How is it that so many of you have developed
negative feelings about your own abilities, despite the fact that
you are all in a college mathematics course and many people don't
make it to this level of course?\textquotedblright{}
\item certain experiences cause new connections in the brain to form or
strengthen, making the brain smarter by literally rewiring it. Here\textquoteright s
some evidence:
\begin{itemize}
\item In a study with rats, researchers put some rats in empty cages and
others in stimulating cages with puzzles and other rats. The rats
in the stimulating environments were smarter, and their brains even
weighed more! 
\item London taxi drivers have to give their brains a workout when they
navigate the complicated streets of London. Research suggests this
has an impact on the brain. The part of the brain responsible for
spatial awareness is bigger in taxi drivers compared to other Londoners.
And the longer a person has been a taxi driver, the bigger that part
of the brain. 
\end{itemize}
\item I don't value students based on their academic performance. I respect
trying and failing at least as much as not trying and succeeding.
\end{itemize}

\subsubsection{Technology policy}
\begin{itemize}
\item No open laptops in class unless I ask you to take them out.
\end{itemize}

\subsubsection{Study groups}

Under ``graded discussions'' in Canvas.

\subsection{Programming}
\begin{itemize}
\item How do professional cryptographers encrypt and decrypt messages?
\begin{itemize}
\item You witnessed how long it took to decrypt messages by hand.
\item Computers can:
\begin{itemize}
\item Try every possible shift in order to break a shift cipher without
even trying.
\item Count the frequency of various encoded (``ciphertext'') letters
in order to determine what the most common letter is, thus telling
you what letter probably corresponds to ``e''
\item By doing this, guess the keyword in the keyword cipher above.
\item Encode messages using a given cipher without any effort on your part.
\end{itemize}
\item For these reasons, we'll be using the programming language Python
in this course.
\begin{itemize}
\item You won't have to come up with code on your own! For everything you'll
need to encode, you'll have a template that you can copy/paste and
just change a few things. \textbf{This is by far the best way to do
programming in this class}.
\item If you have experience programming and/or finish programming early,
you're more than welcome to try to extend what you've done so far!
\end{itemize}
\item Now log in to Codecademy on your computers.
\begin{itemize}
\item ``Introduction to Python'' module; talk students through beginning
example
\item Give class time to work on it ($20$-$30$ min)
\end{itemize}
\end{itemize}
\end{itemize}

\end{document}
