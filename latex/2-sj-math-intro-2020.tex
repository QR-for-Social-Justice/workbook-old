%% LyX 2.3.6.2 created this file.  For more info, see http://www.lyx.org/.
%% Do not edit unless you really know what you are doing.
\documentclass[oneside,english]{amsart}
\usepackage[T1]{fontenc}
\usepackage{amsthm}
\PassOptionsToPackage{normalem}{ulem}
\usepackage{ulem}

\makeatletter
%%%%%%%%%%%%%%%%%%%%%%%%%%%%%% Textclass specific LaTeX commands.
\numberwithin{equation}{section}
\numberwithin{figure}{section}
\theoremstyle{plain}
\newtheorem{thm}{\protect\theoremname}[section]
\theoremstyle{definition}
\newtheorem{xca}[thm]{\protect\exercisename}
\theoremstyle{remark}
\newtheorem{claim}[thm]{\protect\claimname}

\makeatother

\usepackage{babel}
\providecommand{\claimname}{Claim}
\providecommand{\exercisename}{Exercise}
\providecommand{\theoremname}{Theorem}

\begin{document}
\title{WCSBS 220 1 - Course Introduction}
\maketitle

\section{Introduction to the Course}

\subsection{About me}
\begin{itemize}
\item Name/pronouns/PhD from Rice, research in how hard it is to untie mathematical
knots
\begin{itemize}
\item Please feel welcome to talk to me about my research at any time during
the semester if you're interested.
\end{itemize}
\item My primary goal in teaching math is to communicate that math is a
creative, logical field with immense importance to society. which
is like a prosthetic you attach to your reasoning to make it more
powerful.
\end{itemize}

\subsection{Syllabus - read in your own time}

\subsubsection{Intro to Zoom}
\begin{itemize}
\item Breakout Rooms
\item 1-2-3-go
\end{itemize}

\subsubsection{Student introductions}
\begin{itemize}
\item Maddie (she/her): BLM
\item Brianne (she/her): BLM
\item Taylor (they/them): BLM, PoC and trans
\item Claire (Dallasite, she/her): BLM, racism/sexism in professional climbing
\item Jacob (he): BLM, income inequality in schools
\item Grace (she/her): BLM, queer movements
\item Lucy (she/her): performative activism, not actually doing shit
\item Melodey (she/her): persecution of Shi'a Muslims
\item Aleah (she/her): performative activism
\item Tiare (she/her): BLM, stats behind BLM
\end{itemize}

\subsubsection{About the class}
\begin{itemize}
\item In this class, my goal is to get you to stretch your minds by working
in groups on problems that are hard for you. You won't be graded on
how quickly you pick things up or whether you understand every detail,
as long as you develop certain skills that allow you to do math that
is pertinent to your lives.
\item In this course, we'll often follow the following outline:
\begin{enumerate}
\item ask a question or questions related to an issue of social justice
(e.g. are people of color more often exposed to hazardous pollutants
than white people?)
\item delve into some mathematics that will help us answer that question
(e.g. measures of exposure to pollutants, measures of closeness, measures
of ``how PoC-dominant'' a county is)
\item model the social justice question with a mathematical question;
\item use the mathematics we've learned to answer the mathematical question
and interpret the answer in terms of the social justice question.
\end{enumerate}
\item In the syllabus, you'll find information on how to study math, learning
outcomes, textbooks, office hours, and extra credit opportunities.
\end{itemize}

\subsubsection{Office and office hours}
\begin{itemize}
\item My office is Foster 311 (go up the stairs to the third floor, turn
right; I'm at the end of the hall.)
\item My office hours this semester will be MW 1-2pm; TTH 11am-12pm. Anyone
who can't make those hours? I'll usually be around the office 1-4pm
MW, 10am-1pm TTh, and (maybe) 11-noon Friday as well. Make appointments
at least $24$ hours in advance!
\item Tentative course outline: linked in Canvas under Syllabus
\end{itemize}

\subsubsection{Canvas}
\begin{itemize}
\item Sign up now for Canvas notifications
\item I can't guarantee I'll always remember to announce assignments in-class;
you're expected to check Canvas before and after every class period
to determine what assignments you have.
\item I will post all of your grades in Canvas.
\item We take the Title IX and disability provisions extremely seriously!
If you require any academic accommodations related to a disability,
please contact Karen Hicks ASAP.
\end{itemize}

\subsection{What is social justice?}
\begin{xca}
(TPS) What do you think of when you think of social justice? What
are 2-3 issues of SJ you're passionate about?
\begin{itemize}
\item Some possible answers: 
\begin{itemize}
\item justice in terms of the distribution of wealth, opportunities, and
privileges within a society. (Google Dictionary)
\item a concept of fair and just relations between the individual and society.
(Wikipedia)
\item a state or doctrine of egalitarianism (Merriam-Webster)
\item A broad term for action intended to create genuine equality, fairness
and respect among peoples. (University of Massachusetts Lowell Department
of Multicultural Affairs)
\end{itemize}
\item Can we develop a working definition as a class?
\begin{itemize}
\item ``equal access to opportunities''
\item singling out identities vs. broader definition including ``everybody''
\item Equality vs. equity
\end{itemize}
\item One definition (Sensoy-DiAngelo): social justice is a recognition
that:
\begin{itemize}
\item all people are individuals, but we are also members of socially constructed
groups; 
\item society is stratified, and social groups are valued unequally; 
\item social groups that are valued more highly have greater access to resources
and this access is structured into the institutions and cultural norms; 
\item social injustice is real and exists today; 
\item relations of unequal power are constantly being enacted at both the
micro (individual) and macro (structural) levels; 
\item we are all socialized to be complicit in these relations; 
\item those who claim to be for social justice must strategically act from
that claim in ways that challenge social injustice; and 
\item this action requires a commitment to an ongoing and lifelong process. 
\end{itemize}
\item How is this similar to and different from your definitions? The UN's
definition? Thomas Jefferson's?
\end{itemize}
\end{xca}


\subsection{Isn't math ``objective''?}
\begin{itemize}
\item {[}What's wrong with this chart slides{]}
\item https://tinyurl.com/mathoffracking: what do we choose to study as
mathematicians/statisticians?
\item https://tinyurl.com/WMDTalk: Weapons of Math Destruction on the ``value-added
model'' for teacher evaluations
\end{itemize}

\subsection{Introductory problem (TPS in groups)}
\begin{itemize}
\item Let's agree to use classrooms as a measure of wealth, here used as
synonymous with net worth, for today's purposes.
\item For reference, the sum of all US wealth owned by households at the
end of $2016$ was $\$90.2$ trillion\textendash that's $90,200,000,000$.
\end{itemize}
\begin{xca}
First-day activity for a pandemic
\begin{enumerate}
\item \sout{What are some social justice issues that are important to
you as we go into an election in the midst of pandemic, economic hardship,
and what some call a new civil rights movement?}
\item What are some ways we could \textbf{measure} each of these issues?
(how bad they are, solutions, societal reactions, etc.)
\begin{enumerate}
\item Demographics of people who seek mental health services
\item What are costs for therapy, meds, etc.?
\item Income inequality per demographics
\item Level of income at their school
\item Prevalence of COVID-19
\item Healthy Together app
\end{enumerate}
\item What are some things you don't know how to measure about each of these
issues?
\begin{enumerate}
\item asymptomatic COVID cases: how to measure?
\item not all people have access to tests, transportation to test sites
\end{enumerate}
\item Estimate: what are some strategies for measuring these things? Don't
try to plug in numbers yet, just make a list.
\item What are some ways to, as accurately as possible, measure how severe
the COVID-19 outbreak is in Salt Lake City? Write down any assumptions
you make about what ``severe'' means.
\item If you were a disability rights activist concerned about the dangers
of COVID-19 for immunocompromised folks, which of these ways would
you choose to measure? Estimate or find the info.
\item If you were a Westminster administrator, which of these ways would
you choose to measure? Explain, then estimate or find the info.
\item What are some things it would be ideal to be able to measure in order
to curtail or prevent the spread of COVID-19 in SLC? Do the same as
above.
\item \#6 and \#7 for these things.
\item Brainstorm some ways to get the measurements from \#8 and use them
to advocate for positive change.
\item Some ways of measuring whether it would be worthwhile for Westminster
to decrease tuition/fees during COVID-19?
\item Same as above: what would a member of Westminster Student Union (the
independent student activist body) cite? President Dobkin?
\item What questions about social justice, health/accessibility, and math
does this activity bring up for you? I can think of a few:
\begin{enumerate}
\item How did we measure X?
\item What's the best way to measure X?
\item What values are inherent in the way we measure X?
\item If your goal is to advocate for social justice, how does that change
the ``best'' way of measuring X?
\end{enumerate}
\end{enumerate}
\end{xca}

\begin{itemize}
\item We just demonstrated a couple of tools we'll use many times in the
next week or two: 
\begin{itemize}
\item \textbf{Estimation}\textendash round up or down to make calculations
easier and numbers less messy. As we saw when estimating the number
of classrooms needed for the US household wealth, rounding before
the computation ends can introduce error. Therefore, if you have a
calculator, it's best to plug into the calculator before rounding,
whereas if you're computing by hand, rounding may be necessary in
order to increase the accuracy of your computations. Everyone makes
arithmetic mistakes\textendash you'll see me do it plenty of times!
\item \textbf{Dimensional analysis}: start with the given and multiply by
fractions, canceling out the units, until you get something in the
units your answer should be in. This is almost always your answer!
\end{itemize}
\end{itemize}

\section{Setting the Stage}

Get in groups of size 3\textendash 4. Group members should introduce
themselves - name, pronouns, hobbies outside school, goals for the
semester/year, reason they're taking the course. For each of the questions
that follow, I will ask you to: 
\begin{enumerate}
\item Think about a possible answer on your own. 
\item Discuss your answers with the rest of your group. 
\item Share a summary of each group\textquoteright s discussion. 
\end{enumerate}

\subsubsection{Questions}
\begin{enumerate}
\item What are the goals of a liberal arts education? How does social justice
fit in? How does math?
\item How does a person learn something new? 
\item What is the value of making mistakes in the learning process? 
\item How do we create a safe environment where risk taking is encouraged
and productive failure is valued? 
\end{enumerate}
\begin{itemize}
\item I will be teaching this course using IBL methods for all the reasons
you gave, as well as because research shows IBL improves student understanding
of mathematics more than traditional lecture-based teaching methods.
\end{itemize}
\begin{quotation}
\textquotedblleft Any creative endeavor is built on the ash heap of
failure.\textquotedblright \textemdash Mike Starbird
\end{quotation}
\begin{claim}
An education must prepare a student to ask and explore questions in
contexts that do not yet exist. That is, we need individuals capable
of tackling problems they have never encountered and to ask questions
no one has yet thought of. 

If we really want students to be independent, inquisitive, \& persistent,
then we need to provide them with the means to acquire these skills.
\end{claim}


\subsubsection{Engagement Points (Extra Credit Opportunities)}
\begin{itemize}
\item The Westminster College Department of Mathematics values the experiences
that students have with mathematics and data science outside of the
formal classroom. In order to encourage our students to find new and
interesting ways to engage with our disciplines, we have incorporated
a department-wide extra credit policy in which extracurricular activities
gain you \textquotedbl engagement points\textquotedbl{} which will
contribute toward your course grades..
\item Examples of activities that are likely to qualify include, but are
not limited to: attending office hours three times; attending a meeting
of the S-Cubed seminar; volunteering in the East High or Cottonwood
Tutoring Programs; attending Lemma social activities or meetings;
attending college Diversity, Equity, and Inclusion programming; taking
the Putnam exam; attending a screening of a film with mathematical
themes; participating in a Research Experience for Undergraduates
(REU) during the summer; attending a non-required math-related talk
or seminar; attending a regional/national math conference; giving
a talk related to math; submitting or publishing a paper in an academic
journal; participating in a math competition; reading a book with
mathematical themes; and many others.
\item If you participated in an activity that you think qualifies for engagement
points, please fill out the Google Form located at tinyurl.com/WMengagement.
Because we are unable to constantly check the form responses, please
contact me to determine whether you were granted engagement points
for your activity.
\item In this course, you may use three engagement points at any time to
get a free Pass on a reading response or five engagement points to
get a free Pass on a news response. I reserve the right to limit how
many times you may redeem engagement points in these ways. Please
send me an email or tell me in person if you'd like to use your engagement
points.
\end{itemize}

\subsubsection{Tutoring Programs}
\begin{itemize}
\item You may earn engagement points by volunteering your time in Westminster's
Cottonwood High or East High Tutoring Programs. This involves volunteering
at least three times to work at an after-school-tutoring lab at one
of the two high schools with which Westminster has a partnership.
There, you would help high school students struggling with mathematics.
The high-school students receiving this tutoring are usually in Algebra
I, Geometry, or Algebra II and see mathematics as confusing and frustrating.
Both Programs are looking for a willingness to help remove this fear
from a scary subject in their tutors.
\item Students who tutor or TA at least three times in the Cottonwood tutoring
program will receive a stipend of $\$8$ for every hour spent tutoring,
as well as one engagement point per hour volunteered.\} Please see
tinyurl.com/sherlockclub for further details and to sign up for Cottonwood
tutoring or teaching assistantship. 
\end{itemize}

\section{Group work and class norms}

\subsection{Ground rules}

Given the nature of this course, it's important that we develop class
norms to promote an atmosphere which will facilitate the learning
process as well as respect the experiences of different groups in
the classroom and the larger society. The class can agree to revise
them and add others, but all students must commit themselves to the
final set of rules by the end of the first class. These principles
will guide our class discussions and interactions.
\begin{xca}
TPS on what common norms for group work and discussion are important
for our class.
\begin{itemize}
\item Whose voices are valued?
\item What philosophy do we have about who can make contributions to mathematics
and the value of those contributions?
\item How does this connect to the values we espouse for our democracy?
\item 2020 responses:
\begin{itemize}
\item respect (what does this mean?)
\item understanding that everyone's opinions are different
\item openness to hear others' opinions
\item ``treat everyone's opinions equally'' as long as they've not dehumanizing
\item active listening
\begin{itemize}
\item not doing other things on camera
\item not interrupting
\item interact once they're done talking
\end{itemize}
\end{itemize}
\end{itemize}
\end{xca}

Here are some useful classroom norms:
\begin{enumerate}
\item \textbf{Controversy and vulnerability}: different views are expected
and honored with a group commitment to understand the sources of disagreement
and to work cooperatively toward common solutions \textbf{without
putting a disproportionate responsibility on marginalized groups to
educate us}. 
\begin{itemize}
\item No one's identities are under question: e.g. no arguing that queer
people are wrong for existing.
\item Please come talk to me anytime you feel uncomfortable enough that
it's affecting your work in class. 
\end{itemize}
\item \textbf{Own your intentions and your impact. }Acknowledge that the
impact of your actions is not always congruent with your intentions
and that positive or neutral intentions do not trump negative impact.
Participate truthfully and accept when your actions or words harm
others.
\item \textbf{Challenge by choice}: individuals may determine for themselves
if and to what degree they will participate in a given discussion
or activity. It is possible not to participate in a few activities
and still receive an A in the course. Be aware of what factors influence
your decision about whether to challenge yourself on a given issue.
\item \textbf{Respect}: each other and yourselves. Investigate how your
cultural context affects what you think of as respect for others and
acknowledge what respect looks like in other contexts. 
\begin{xca}
(TPS) What does respect look like for you? How might ideas of respect
vary with cultural context? How might you firmly challenge the views
of someone else in a respectful manner?
\end{xca}

\item \textbf{No attacks}. personal attacks are a form of extreme disrespect.
Disagreeing with ideas is welcome; attacking individuals for their
ideas is not. 
\begin{xca}
What are the differences betweeen a personal attack and a challenge
to an idea that makes an individual feel uncomfortable? What are some
situations that might blur the lines between the two? How can we acknowledge
when certain beliefs (e.g. trans people not existing) are inherently
personal attacks?
\end{xca}

\end{enumerate}
\begin{itemize}
\item Group roles: In order to facilitate group interaction, you will take
on group roles designed to mimic the role of mathematical and/or sociological
researchers. The roles are as follows \textbf{(follow along in your
syllabus)}:
\begin{itemize}
\item \textbf{The facilitator} is responsible for making sure every student
is able to contribute and be heard. \emph{Contributions} may include
asking good questions, rephrasing someone else's idea, coming up with
a way of connecting mathematics to the real world, and many others. 
\item the \textbf{resource manager} is responsible for obtaining and keeping
track of all necessary resources to solve a problem. \emph{Resources}
may include writing utensils, paper, the Internet, your instructor,
data sources, and most importantly, your team. 
\item the \textbf{lead author} is responsible for writing down the ideas
that each group comes up with. 
\item the \textbf{communicator} is responsible for reporting what your group
came up with to the class, instructor, and any relevant community
groups. 
\end{itemize}
\item Assign roles to each group by first letters of first name; they'll
rotate every day of class.
\item Anyone can answer questions posed to the whole class, but if your
group came up with an idea th
\item at hasn't been shared yet, it \textbf{someone} in your group's responsibility
to share your answer so that everyone can learn from you!
\end{itemize}

\subsection{Homework}
\begin{enumerate}
\item Vote on a social justice topic of your choice to discuss starting
Week 3 of the course.
\item Write in a topic that you'd like to discuss in this class, whether
it's already on the list or not.
\item Electronically sign the learning contract for the course (5 min)
\item Read the syllabus (by Tuesday) and bring any questions about it to
class.
\item (by Friday of next week, but start working now) Write a 1-2 page (double
spaced) mathematical autobiography based on the questions on Canvas
and turn it in to me in my office, either during office hours or email
me for an appointment.
\end{enumerate}

\end{document}
