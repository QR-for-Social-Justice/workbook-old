%% LyX 2.3.6.2 created this file.  For more info, see http://www.lyx.org/.
%% Do not edit unless you really know what you are doing.
\documentclass[oneside,english]{amsart}
\usepackage[T1]{fontenc}
\usepackage{geometry}
\geometry{verbose,tmargin=1in,bmargin=1in,lmargin=1in,rmargin=1in}
\usepackage{url}
\usepackage{amsthm}
\usepackage{graphicx}
\PassOptionsToPackage{normalem}{ulem}
\usepackage{ulem}

\makeatletter

%%%%%%%%%%%%%%%%%%%%%%%%%%%%%% LyX specific LaTeX commands.
%% Because html converters don't know tabularnewline
\providecommand{\tabularnewline}{\\}

%%%%%%%%%%%%%%%%%%%%%%%%%%%%%% Textclass specific LaTeX commands.
\numberwithin{equation}{section}
\numberwithin{figure}{section}
\theoremstyle{plain}
\newtheorem{thm}{\protect\theoremname}
\theoremstyle{definition}
\newtheorem{xca}[thm]{\protect\exercisename}
\theoremstyle{plain}
\newtheorem{question}[thm]{\protect\questionname}

\makeatother

\usepackage{babel}
\providecommand{\exercisename}{Exercise}
\providecommand{\questionname}{Question}
\providecommand{\theoremname}{Theorem}

\begin{document}
\title{WCSAM 206 7 - Le Chiffre Indechiffrable (``The Indecipherable Cipher'')}
\maketitle

\section{Python module: Frequency Analysis}

\textbf{Solving Affine Ciphers}:
\begin{enumerate}
\item Form a system of linear congruences to solve for $m$, the multiplicative
constant, and $k$, the additive constant.
\item Simplify the system of congruences by substitution, elimination, or
addition/subtraction of congruences.
\item Find the solution of the resulting congruence using trial and error
or the extended Euclidean algorithm. 
\item Use $m$ and $k$ to decrypt the ciphertext using the formula $p_{i}=m^{-1}(c_{i}-k)$
mod $26$.
\end{enumerate}

\subsection{Python Module: Frequency Analysis}
\begin{itemize}
\item In order to break an affine (or any monoalphabetic) cipher, we have
to do a frequency count.
\item For longer messages, it's very impractical to frequency analyze by
hand. For example, consider what would happen if you intercepted the
entire Declaration of Independence enciphered with $C\equiv25P+5\mod26$.
\item Let's start to build a tool to frequency analyze in Python!
\item The algorithm we'll use is as follows: {[}show what would happen in
the message ``TEST''{]}
\begin{itemize}
\item \textbf{Goal}: build a list of ciphertext letters together with their
corresponding counts in the message.
\begin{enumerate}
\item Start with an empty count\_list.
\item Add 26 zeroes to the empty count\_list, one to represent each ciphertext
letter.
\item Go through the intercepted ciphertext message letter by letter and,
for each ciphertext letter in the message, add $1$ to that letter's
entry in count\_list.
\item Label each entry of count\_list with the corresponding ciphertext
letter.
\item Return the labeled count\_list to the user.
\end{enumerate}
\end{itemize}
\end{itemize}
\begin{xca}
{[}CoCalc{]} open the Frequency Analysis module and work through it,
step-by-step, with the class, giving them occasional tasks.
\end{xca}


\section{``The Indecipherable cipher''}

\textbf{Last time: }

\subsection*{The Process of Frequency Analysis}
\begin{enumerate}
\item Count the number of each letter in the ciphertext.
\item Guess that the most common letter is equal to \textbf{e}. This determines
the key. Attempt to decrypt this message using that key
\item If the result is unintelligible, guess that the most common letter
is equal to \textbf{t}. Repeat. If this doesn't work, guess that the
most common letter is equal to \textbf{a}, etc. Use your knowledge
of English to help if possible.
\end{enumerate}
\textbf{Today}: frequency analysis broke the simple substitution cipher
and let cryptanalysts read almost any encrypted message they wanted.
Until the development of a new cipher...
\begin{itemize}
\item Re-shuffle groups
\end{itemize}
\begin{xca}
\textbf{(reading question) }Pass your ``alternating shift'' ciphertext
clockwise in your groups and have your classmates try to break it
without knowing the key. Try a quick frequency count on the ciphertext
you receive to see if frequency analysis will help, and feel free
to try and other tricks to decrypt the message aside from asking for
the key.
\begin{itemize}
\item None of these probably worked...the ciphertext is so short that it's
not much longer than the keyword! This is approaching a \textbf{one-time
pad, }which is theoretically unbreakable.
\end{itemize}
\end{xca}

~
\begin{xca}
\textbf{(RQ) }Watch the Numberphile video about the flaw in the Enigma
machine, then answer the following questions in full sentences, explaining
your reasoning in each case.
\begin{enumerate}
\item Does the Enigma machine encrypt using a monoalphabetic or polyalphabetic
substitution cipher? Explain your answer. 
\begin{itemize}
\item The Enigma machine uses a polyalphabetic cipher, because the same
letter can be enciphered in multiple different ways depending on the
current rotor settings.
\end{itemize}
\item What mechanical aspect(s) of the Enigma machine make it monoalphabetic
or polyalphabetic?
\begin{itemize}
\item The rotors rotate, effectively changing the cipher alphabet every
time a new letter is encrypted.
\end{itemize}
\item What was the major flaw in the Enigma machine that allowed its codes
to be broken? Why was this a flaw? 
\begin{itemize}
\item No letter could be encrypted as itself. This was a flaw because we
can eliminate certain letter/word possibilties simply by determining
if any letters would be encrypted as themselves.
\end{itemize}
\item Describe in 3-4 sentences how the Bombe machine (called \textquotedbl Christopher\textquotedbl{}
in The Imitation Game) tests rotor positions in order to guess the
settings of the Enigma machine. 
\begin{itemize}
\item The Bombe took common German phrases found in Enigma plaintext, such
as ``wetterbericht'' (``weather report''), and slid it along the
Enigma ciphertext until it determined where in the ciphertext the
plaintext phrase could possibly occur.
\item The Bombe machine tried to work out the plugboard at the front of
the machine, which transposes (switches) pairs of letters. 
\item Guessing a pair of letters which are connected in the plugboard allows
us to determine other plugboard settings, since the way the rotors
operate are predictable and change in a predictable way.
\item Assuming, e.g., $(ta)$ is a plugboard setting (in other words, $t$
and $a$ are swapped), if we conclude that $(tx)$ is a plugboard
setting, this is a \textbf{contradiction }because $t$ can only be
connected to at most one other letter in the plugboard.
\item Once we find a contradiction, we can reject all the conclusions we
came to from our initial plugboard guess because they're ``fruit
of a poison tree''.
\end{itemize}
\item What made naval Enigma codes harder to break than Army or Air Force
codes? 
\begin{itemize}
\item Naval Enigma codes had an extra rotor, making four rotors total instead
of three.
\end{itemize}
\item If you had to explain to someone who knew no cryptography or probability
how the Enigma code was broken, what would you say?
\begin{itemize}
\item The Bombe machine begins by assuming two letters are connected on
the plugboard, hence are switched with each other. 
\item It then uses known phrases or words and the fact that no letter is
connected to itself to deduce other rotor positions from the initial
guess.
\item If we get a contradiction (a given letter being connected to more
than one other letter), then we can eliminate all the deductions we
made earlier.
\item Once we eliminate all but one setting, that setting can be input into
a captured or reverse-engineered Enigma machine to decrypt messages.
\end{itemize}
\end{enumerate}
\end{xca}


\section{Response to anonymous evals}
\begin{itemize}
\item Discuss what we want each function to do and steps (in Python) to
do it before each module 
\item Homework not properly synced with class material 
\item Cut down on number of assigments
\end{itemize}

\section*{The Vigenere Cipher}
\begin{xca}
First, read p66-67 of The Code Book, through the paragraph ending
in \textquotedbl smooth\textquotedbl , on Charles Babbage and his
cryptanalysis of the Vigenere cipher. Then encrypt the following messages
with the corresponding Vigen\`{e}re keywords, showing all your work,
and bring your work to class to turn in on the date below. You may
want to use the Vigen\`{e}re square on p102 of your CoursePack or
modular arithmetic. Finally, answer the question below the messages.
\begin{enumerate}
\item \textquotedbl checkmate\textquotedbl , keyword = \textquotedbl rook\textquotedbl{} 
\begin{enumerate}
\item TVSMBAODV
\end{enumerate}
\item \textquotedbl poison the river\textquotedbl , keyword = \textquotedbl kafka\textquotedbl{}
\begin{enumerate}
\item zoncox tmo rsvjb 
\end{enumerate}
\item \textquotedbl the sun and the man in the moon\textquotedbl , keyword
= \textquotedbl king\textquotedbl{} 
\begin{enumerate}
\item dpr yev ntn buk wia ox buk wwbt 
\end{enumerate}
\end{enumerate}
In message 3, you may have noticed that \textquotedbl the\textquotedbl{}
was enciphered in the same way twice. Why did this happen? If you
intercepted the ciphertext from question 3 and wanted to guess the
length of the key used to encrypt the message, can you think of a
way to use this repetition to guess the key? Explain your answer.
\begin{itemize}
\item Since the ``BUK''s in the ciphertext are separated by $8$ letters
(start at the first b, which we call $0$, and count up), we guess
that the keyword must repeat at least every $8$ letters.
\item For it to loop around like that every $8$ letters, the keyword either
has to be $8$ letters long or a factor of $8$ (for example, if it's
$4$ letters long $C_{1}C_{2}C_{3}C_{4}$, then the first b could
fall on $C_{1}$ and the other characters
\begin{align*}
P_{1}P_{2}P_{3}P_{4}P_{5}P_{6}P_{7}P_{8}\\
K_{1}K_{2}K_{3}K_{4}K_{1}K_{2}K_{3}K_{4}
\end{align*}
\item Any number that goes into $8$ evenly (any factor of $8$) would loop
back around to ensure that $P_{1}$ and $P_{8}$ are encrypted with
$K_{1}$.
\end{itemize}
\end{xca}

\begin{itemize}
\item It turns out the Enigma machine uses a souped-up version of a Vigen\`{e}re
cipher, called a \emph{one-time pad}. So to understand Enigma, we
need to understand Vigen\`{e}re first.
\end{itemize}
The Vigenere cipher uses not one, but 26 distinct cipher alphabets
to encrypt a message. 
\begin{itemize}
\item {[}slide{]} First look at a Vigenere square, like the one in Table
3 of Singh p48 or p95 (bottom numbering) of your CoursePack:

\includegraphics{\string"Old Cryptography Notes/pasted8\string".png}
\item Row 1 represents a cipher alphabet with a Caesar shift of 1, meaning
it could be used to implement a Caesar shift cipher in which every
letter of the plaintext is replaced by the letter one place further
on in the alphabet. Similarly, row 2 represents a cipher alphabet
with a Caesar shift of 2, and so on.
\item We use a diferent row of the Vigenere square (a different cipher alphabet)
to encrypt different letters of the message.
\item The sender and receiver must agree on a system of switching between
rows. This is achieved by using a keyword. To illustrate, let's encipher
\textbf{divert troops to east ridge} using the keyword \textbf{WHITE}.
\end{itemize}
\begin{enumerate}
\item First off, the keyword is spelled out above the message, and repeated
over and over again so that each letter in the message is associated
with a letter from the keyword: 

\includegraphics{\string"Old Cryptography Notes/pasted9\string".png}
\item To encrypt the first letter, begin by identifying the letter above
it, which in turn defines a particular row in the Vigenere square.
Look where the column headed by the plaintext letter intersects the
row corresponding to the first keyword letter.

The first letter, \textbf{d}, has \textbf{W} above it, so we look
at row 22 of the Vigenere square to encipher it. We look where the
column headed by \textbf{d} intersects the row beginning with \textbf{W}.
We get the letter \textbf{Z. }Hence, the letter \textbf{d }in the
plaintext is represented by \textbf{Z }in the ciphertext.
\item Continue this process for the rest of the letters.

In your groups, finish enciphering \textbf{divert troops to east ridge}.

\includegraphics{\string"Old Cryptography Notes/pasted10\string".png}

\end{enumerate}
\begin{xca}
Encipher the lyrics ``It's Friday, Friday, gotta get down on Friday''
using the keyword ``Terrible''. How does it mask the repetition
of the word Friday?

Encipher the lyrics ``She's just a small town girl'' using the keyword
``Journey''. Now use the keyword ``Don't Stop Believing''. Which
offers more security?

\includegraphics{\string"Old Cryptography Notes/pasted127\string".png}
\end{xca}

\begin{itemize}
\item {[}slide{]} Why is the Vigen\`{e}re cipher so strong?
\begin{itemize}
\item Number of possible ciphers (the keyspace) is huge! For each letter
of the keyword, there are $26$ choices. For a keyword that's only
six letters long, that's 
\[
26\times26\times26\times26\times26\times26=26^{6}=308,915,776
\]
 possibilities.
\item Letter frequencies are masked because all the e's in the plaintext
might go to different ciphertext letters, making it useless to count
the most common letter and guess it came from e.
\end{itemize}
\end{itemize}
\begin{xca}
{[}slide{]} Decrypt the message encrypted with Vigenere keyword ``SIKEN''.
\end{xca}

\begin{question}
What other advantages does the Vigenere cipher have over the shift
cipher?
\begin{itemize}
\item It may be easier to remember a keyword like ``BONANZA'' than a number
shift, like $21$\textendash much less several numerical shifts!
\item Breaking (or even decrypting) a Vigen\`{e}re-enciphered message requires
more work and more knowledge of mathematics than decrypting a shift
cipher.
\end{itemize}
\end{question}

\begin{itemize}
\item The Vigenere cipher was not immediately adopted because it took a
long time to encode things, but when the telegraph was invented it
gained greatly in popularity. It was considered unbreakable, and called
in French \emph{le chiffre ind\'{e}chiffrable}, the ``indecipherable
cipher''. But it wasn't really indecipherable...
\end{itemize}

\subsection{Reading Question: the Development of Cipher Machines}
\begin{xca}
Please read p124-142 of The Code Book (under the heading The Development
of Cipher Machines), then answer the following questions in a text
box below:
\begin{enumerate}
\item Describe the role of the scramblers in the Enigma machine. What is
the benefit of adding multiple scramblers instead of just one? 
\begin{enumerate}
\item The scramblers encipher individual letters by connecting them via
wires. For instance, plaintext a might go to A in the scrambler, which
is connected by a wire to B, and the current goes out of the scrambler
into the B lamp.
\item Having multiple scramblers allows the cipher alphabet to change after
each letter is enciphered, since scramblers rotate each other after
encryption. Moreover, even after typing 26 letters, the cipher alphabet
won't go back to the beginning because the second rotor will rotate.
\end{enumerate}
\item What is the benefit of adding a reflector to the Enigma machine? 
\begin{enumerate}
\item The reflector sends the current back through the machine in a mirror-symmetric
way.
\item This means that to get plaintext back from ciphertext, all you have
to do is run the ciphertext through the machine.
\item Encryption and decryption are the same process!
\end{enumerate}
\item Name two additional Enigma features that made decrypting Enigma more
difficult. How did these features add to the security of the Enigma
machine?
\begin{enumerate}
\item The plugboard switched letters with each other to add to the number
of possible encryptions.
\item The scramblers are removable and interchangeable, increasing the number
of keys in a 3-rotor Enigma by a factor of $3!=6$.
\end{enumerate}
\end{enumerate}
\end{xca}


\subsection{Charles Babbage: Finding the Key Length for the Vigenere Cipher}

Let's say you're handed a cipher and know it's a Vigenere cipher.
{[}This is not very realistic; it's unlikely someone will have told
you what kind of cipher you've intercepted! We'll talk about ways
to figure out what cipher you're dealing with later.{]} You don't
know the length of the key or anything about the original message.

Charles Babbage figured out how to solve the Vigenere cipher by thinking
very hard about examples, among other things. 

\includegraphics{\string"Old Cryptography Notes/pasted22\string".png}

\includegraphics{\string"Old Cryptography Notes/pasted23\string".png}
\begin{xca}
First, read p66-67 of The Code Book, through the paragraph ending
in \textquotedbl smooth\textquotedbl , on Charles Babbage and his
cryptanalysis of the Vigenere cipher. Then encrypt the following messages
with the corresponding Vigen\`{e}re keywords, showing all your work,
and bring your work to class to turn in on the date below. You may
want to use the Vigen\`{e}re square on p102 of your CoursePack or
modular arithmetic. Finally, answer the question below the messages.
\begin{enumerate}
\item \textquotedbl checkmate\textquotedbl , keyword = \textquotedbl rook\textquotedbl{} 
\item \textquotedbl poison the river\textquotedbl , keyword = \textquotedbl kafka\textquotedbl{} 
\item \textquotedbl the sun and the man in the moon\textquotedbl , keyword
= \textquotedbl king\textquotedbl{} 
\end{enumerate}
In message 3, you may have noticed that \textquotedbl the\textquotedbl{}
was enciphered in the same way twice. Why did this happen? If you
intercepted the ciphertext from question 3 and wanted to guess the
length of the key used to encrypt the message, can you think of a
way to use this repetition to guess the key? Explain your answer.

\includegraphics{\string"Old Cryptography Notes/pasted14\string".PNG}
\begin{itemize}
\item The word \textbf{the} is enciphered as \textbf{DPR }in the first instance,
and then as \textbf{BUK }on the second and third occasions. This is
because the second \textbf{the} is displaced by 8 letters from the
third \textbf{the}, and eight is a multiple of the length of the keyword,
which is four letters long. In other words, the second \textbf{the}
is directly below \textbf{ING}, and the third \textbf{the} falls in
the same place!
\end{itemize}
\end{xca}

\begin{xca}
\textbf{{[}RQ{]}}: 
\begin{enumerate}
\item In your own words, describe (step by step) the method Charles Babbage
used to find the length of the key in a Vigenere cipher. 
\item Suppose you intercept a Vigenere-enciphered message which has the
ciphertext letters Y-P-I-V repeated 35 characters apart and the letters
X-I-V-Q-Q repeated 56 characters apart. How long would you guess the
key is? Explain your answer. 
\begin{enumerate}
\item the key length could be any of the shared factors of $35$ and $56$. 
\item Since $7$ goes evenly into $35=7\times5$ and $56=7\times8$, and
their only other shared factor is $1$, the keyword is almost certainly
$7$ letters long.
\end{enumerate}
\item After you have determined the key length of a Vigenere cipher, do
you have any thoughts on how you could use frequency analysis to determine
the keyword (and thus decrypt the message)?
\begin{enumerate}
\item Since we know every seventh ciphertext letter (in the example above)
was enciphered with the same keyword letter, we can do frequency analysis
on every seventh letter in order to figure out the first letter of
the keyword.
\item Then, we could analyze the second letter, the eighth letter, and every
letter congruent to $1\mod7$ to figure out the second letter of the
keyword.
\item Continue this process to find the entire keyword.
\item How do we decrypt a Vigenere-enciphered message once we know the keyword?
Share methods using the square and modular arithmetic. Since we're
already using modular arithmetic to find the keyword, we might as
well use it to decrypt the message!
\end{enumerate}
\end{enumerate}
\end{xca}

\begin{itemize}
\item {[}slide{]} Now let's talk about Babbage's method. We'll use this
ciphertext as an example:

\includegraphics{\string"Old Cryptography Notes/pasted11\string".png}
\end{itemize}
\begin{enumerate}
\item The first step is to find all sequences of four or more letters that
are repeated more than once in the ciphertext.

{[}slide{]} The sequence \textbf{E-F-I-Q }is repeated in the first
line and in the fifth line, shifted forward by 95 letters. Why else
could this happen?\emph{ }

\emph{Well, most likely the same sequence of letters in the plaintext
has been enciphered using the same part of the key. Alternatively,
there's a slight chance that two different sequences of letters in
the plaintext got enciphered using different parts of the text and
coincidentally led to the identical sequence in the ciphertext. We'll
restrict to sequences of four or more letters so the second possibility
is slight.}

If we could figure out the keyword, we'd be done, since it's used
by the receiver to decipher the message. We can't yet figure out the
keyword, but we can figure out its length.
\item For each repeated sequence, count the displacement/spacing between
the first repeat and the second repeat. Write it down in a table.

Here's a table for this ciphertext:

\includegraphics{\string"Old Cryptography Notes/pasted15\string".png}

Remember to start counting the spacing at the first letter and end
at the first letter of the next repeat. For instance, for \textbf{E-F-I-Q},
count 1 for \textbf{F}, 2 for \textbf{I}, and so on until you reach
the \textbf{E }of the next \textbf{E-F-I-Q}.
\item Write out the factors, or divisors, of each of the numbers in the
``repeat spacing'' column.

Y'all can do this part!
\begin{itemize}
\item The factors of 95 are 1, 5, 19, and 95.
\item For 5, the only factors are 5 and 1, since 5 is prime.
\item The factors of 20 are 1, 2, 4, 5, 10, and 20.
\item The factors of 120 are 2, 3, 4, 5, 6, 8, 10, 12, 15, 20, ...
\end{itemize}
\begin{question}
So what are our possible key lengths? Is it likely that the keyword
is more than 20 letters long? How about 1 letter long? Why or why
not?
\end{question}

\begin{itemize}
\item Well, most English words are less than 20 letters long, so we'll assume
the key is between 2 and 20 letters long. A 1-letter key is just a
shift cipher, not a real Vigenere cipher.
\item What other possible key lengths are there?
\item Could the key be 3 letters long? Why or why not?
\end{itemize}
Well, if the key were 3 letters long, call the first letter \textbf{$\mathbf{L_{1}}$},
the second letter $\mathbf{L_{2}}$, and the third letter $\mathbf{L_{3}}$,
then we'd have the following situation:

\begin{tabular}{|c|c|c|c|c|c|c|c|c|c|c|c|c|c|c|c|c|c|c|c|c|c|c|c|c|}
\hline 
$\mathbf{L_{1}}$ & $\mathbf{L_{2}}$ & $\mathbf{L_{3}}$ & $\mathbf{L_{1}}$ & $\mathbf{L_{2}}$ & $\mathbf{L_{3}}$ & $\mathbf{L_{1}}$ & $\mathbf{L_{2}}$ & $\mathbf{L_{3}}$ & $\mathbf{L_{1}}$ & $\mathbf{L_{2}}$ & $\mathbf{L_{3}}$ & $\mathbf{L_{1}}$ & $\mathbf{L_{2}}$ & $\mathbf{L_{3}}$ & $\mathbf{L_{1}}$ & $\mathbf{L_{2}}$ & $\mathbf{L_{3}}$ & $\mathbf{L_{1}}$ & $\mathbf{L_{2}}$ & $\mathbf{L_{3}}$ & $\mathbf{L_{1}}$ & $\mathbf{L_{2}}$ & $\mathbf{L_{3}}$ & $\mathbf{L_{1}}$\tabularnewline
\hline 
\hline 
\textbf{W} & \textbf{C} & \textbf{X} & \textbf{Y} & \textbf{M} & \textbf{?} & \textbf{?} & \textbf{?} & \textbf{?} & ? & \textbf{?} & \textbf{?} & \textbf{?} & \textbf{?} & ? & \textbf{?} & \textbf{?} & \textbf{?} & \textbf{?} & ? & \textbf{W} & \textbf{C} & \textbf{X} & \textbf{Y} & \textbf{M}\tabularnewline
\hline 
\end{tabular}

The first \textbf{W }would have been enciphered by the $\mathbf{L_{1}}$-alphabet,
and the second \textbf{W }by the $\mathbf{L_{3}}$-alphabet. It's
really unlikely that the repeat would happen like this! 

It looks like we'd need the initial \textbf{W }of the second \textbf{W-C-X-Y-M
}to be on \textbf{$\mathbf{L_{1}}$ }if the initial \textbf{W }of
the first \textbf{W-C-X-Y-M }was. That means the first row would have
to start over every 20 letters. 

What possible length keywords could we have now?

There are 6 possibilities:

\includegraphics{\string"Old Cryptography Notes/pasted16\string".png}

So the length of the key has to be one of the \textbf{factors} of
20!
\item The possible key lengths are the common factors of all of the repeat
spacings.

The same thing we did in step 3 applies to all the other repeat lengths:
95, 5, and 120. The keyword length must divide each of 95, 5, 20,
and 120. The only numbers that do this are 1 and 5, and we already
said 1 couldn't be the length. So the length of the keyword is 5.

\includegraphics{\string"Old Cryptography Notes/pasted17\string".png}

\includegraphics{\string"Old Cryptography Notes/pasted18\string".png}

\end{enumerate}
\begin{xca}
{[}slide; print{]} Determine the length of the key for the following
Vigenere text using the underlined words:

oiggbwagnuzsbyhtwsqpammvxiggoufwnleqaigvqnrdantdwntxbwacccztwnmcnqzpccaclizrnchtmczarvqgcsmcmxqsrwminxfdcbqeaibdbcfxxhfwjnmaugqcjlqrayminxqfduxcxqitjlqtwamvnxucjadtjnoxecxljlftbnucpqttcbqgcbmiwufxxhagjhkcjnudwmarxhotrpqsjhphxxqsrwminxopwfacpyzsdlqlnudtvyfdwusgnufqjnfanzutuxaucbmifudlnbmknwabnnasnxurjnqpyidirizdontpczutuxmhjzucjfdtbnucpjxplyrdantdbyiwxbqgnamknnttrlxxeye\uline{iqufiquf}cjnudwguvqnxxeyuirmmacistcbqgocfirhspwxbgxjqgcbmifyewxoxsmifwrmnjcczpuudvnletwmqlnwmcwifsnxurjnqlnwmcwifrxhetllminqqrjhzdcbmauiiiqcevaigcmnttklmkngqcuchxwamcmxqpmqtdbndjpaxtmbqgnbmknwacbyogjnqsrnrpaundeyajajadajalnlfdjxpdaxqiauoicbqlxlxsfcxaucfiuyzdcyzdafacplqbngntaqtpcqqhjsttaynjccfrjhzteyduxlstcqtpcntthxusqydtrnuhoidjbnttuchxwadpcbqgcintmypxluftmbqgnnaiqygcoczxbbqsfidzfburqntthqtdoigvqnttaytpeyfwdmrpamacxvxnjxhpwwqsrnuhaufwnlrdaoeixvqwnlqsnxurjnqscifwnadtjnfpbedtvuucrhsqnzagnoeiqufuaiyiqyetqizdaypsnuplnnmznczraymhnxpteifxxhfdcbmilughnzagfburqntthamknnttuueiooxavymhdlqdoxqkxnudwntpcqqwnlqwratahlqhxfhtcbmicbqhnxqpmmtpufzdcbmknxutmczkjcziqufiqcecjnudwozsnlsdmmtpuftpeymcnqnxantdozdtnxabjhpiqufvxpqgwgqcciriqybtxjxtksfwnjqdyfquxlfwnjqdyfqhquxawifenluhqzddvnttnudiq
\begin{itemize}
\item The spacing is $4$, and the factors of the spacing are $1,2,4$.
Since $1$ and $2$ are very unlikely lengths for a Vigen\`{e}re cipher,
since the code-maker wants more security than that if they're bothering
to use Vigen\`{e}re to begin with, the key length is almost certainly
$4$.
\end{itemize}
\end{xca}

\begin{question}
After we determine the length of the keyword, how can we use frequency
analysis to decrypt the message using the repetition of the keyword
in the cipher?
\end{question}

\begin{xca}
{[}slide{]} Have students submit Vigen\`{e}re ciphertext with a {*}long{*}
phrase as plaintext and a 3-8 letter keyword. Then assign people to
try and find repetitions, if possible, to crack the length of the
key. Then use https://tinyurl.com/SinghTool can decrypt automatically
based on guessing key length and using frequency analysis on every
$n$th letter, where $n$ is the length of the key.
\end{xca}

~
\begin{xca}
0.3, 2.14 in Coursepack, 2.4(a-b), find the possible key length(s)
for the following Vigenere ciphertext:

\includegraphics{\string"Old Cryptography Notes/pasted20\string".png}

and crack it using frequency analysis. This tool: http://www.counton.org/explorer/codebreaking/frequency-analysis.php
can help.
\end{xca}


\section*{Kryptos}
\begin{itemize}
\item {[}slide{]} Part 1: \url{https://www.youtube.com/watch?v=PuTddBGq3e0}
\item Part 2: \url{https://www.youtube.com/watch?v=eBlApaPx3xU}
\end{itemize}
In fact, Kryptos (parts 1 and 2) used a modified Vigenere square.
In order to find the length of the keywords (there were two!) there,
the solvers had to use something called the Index of Coincidence,
which we'll learn next week.

Here's a quote from the CIA guy who first solved Parts I-III, in 1998:
``Nothing is more frustrating than reading a cryptography book where
the author easily and in a straightforward manner shows how the code
was solved. It is a little like reading one of those books on \textquotedbl How
I made a million dollars.\textquotedbl{} The methods typically work
for that particular set of circumstances, but they often do not work
in your particular case. Similarly, codes have distinct characteristics
that frequently require each to be solved with a unique method. Because
I did not see any need to dwell on the hundreds of things I had tried
that didn\textquoteright t work, the methodologies I presented may
seem a little too straightforward and deceptively easy. As the adage
goes, \textquotedbl If at first you don\textquoteright t succeed\textquotedbl\textemdash and
in cryptology you never do\textemdash \textquotedbl try, try again.\textquotedbl{}
If the methods that I tried had not worked, I would have tried others
until I found something that did.

When confronted with a puzzle or problem, we sometimes can lose sight
of the fact that we have issued a challenge to ourselves, not to our
tools. And before we automatically reach for our computers, we sometimes
need to remember that we already possess the most essential and powerful
problem-solving tool within our own minds.''

\section*{The Thomas Beale Cipher}

http://www.thomasbealecipher.com/

Try to find and decrypt the hidden messages that hold clues to the
characters' secrets!

\section*{Read ``Cracking the Enigma'' in The Code Book}

\section*{Enigma Machine Videos: Numberphile}

https://www.youtube.com/watch?v=G2\_Q9FoD-oQ\&src\_vid=V4V2bpZlqx8\&feature=iv\&annotation\_id=annotation\_309534

\section*{Sample Enigma Encoding}

http://enigma.louisedade.co.uk/
\end{document}
