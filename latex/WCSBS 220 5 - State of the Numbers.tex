%% LyX 2.3.6.2 created this file.  For more info, see http://www.lyx.org/.
%% Do not edit unless you really know what you are doing.
\documentclass[oneside,english]{amsart}
\usepackage[T1]{fontenc}
\usepackage{geometry}
\geometry{verbose,tmargin=1in,bmargin=1in,lmargin=1in,rmargin=1in}
\usepackage{babel}
\usepackage{amsthm}
\usepackage[unicode=true]
 {hyperref}

\makeatletter
%%%%%%%%%%%%%%%%%%%%%%%%%%%%%% Textclass specific LaTeX commands.
\numberwithin{equation}{section}
\numberwithin{figure}{section}
\theoremstyle{plain}
\newtheorem{thm}{\protect\theoremname}
\theoremstyle{definition}
\newtheorem{xca}[thm]{\protect\exercisename}
\theoremstyle{definition}
\newtheorem{example}[thm]{\protect\examplename}
\theoremstyle{definition}
\newtheorem{defn}[thm]{\protect\definitionname}

\makeatother

\providecommand{\definitionname}{Definition}
\providecommand{\examplename}{Example}
\providecommand{\exercisename}{Exercise}
\providecommand{\theoremname}{Theorem}

\begin{document}
\title{WCSBS 5 - State of the Numbers}
\maketitle

\section{Reading questions: SOTU Special}
\begin{xca}
\textbf{(Reading Question) }On January 30, 2018, the 45th President
of the United States of America, Donald Trump, gave the Constitution-mandated
State of the Union. For each, get skeptical by doing the following,
writing your work on a sheet of paper, and bringing it to class on
the date below.
\begin{itemize}
\item Before pulling out an electronic device, write a list of 1-2 questions
you\textquoteright d like to know about the numbers\textendash things
that would help you understand if the numbers are big or small, and
whether or not the underlying data actually supports the President\textquoteright s
argument. 
\item Get online and try to answer your questions. Rate the President\textquoteright s
claims on two scales: 
\begin{itemize}
\item How true is the statement? (0 = completely false, 4 = completely true) 
\item How misleading is the statement? (0 = not at all, 4 = totally misleading)
(In other words, when you look at the underlying data, does it support
the President\textquoteright s point or not?)
\end{itemize}
\end{itemize}
\begin{enumerate}
\item Unemployment claims have hit a 45-year low. African-American unemployment
stands at the lowest rate ever recorded, and Hispanic American unemployment
has also reached the lowest levels in history. 
\item In 2016, we lost 64,000 Americans to drug overdoses: 174 deaths per
day. Seven per hour. We must get much tougher on drug dealers and
pushers if we are going to succeed in stopping this scourge. 
\item Since we passed tax cuts, roughly 3 million workers have already gotten
tax cut bonuses -{}- many of them thousands of dollars per worker.
Apple has just announced it plans to invest a total of \$350 billion
in America, and hire another 20,000 workers. 
\item The first pillar of our framework generously offers a path to citizenship
for 1.8 million illegal immigrants who were brought here by their
parents at a young age -{}- that covers almost three times more people
than the previous administration.
\end{enumerate}
\end{xca}


\section{Reading Questions: WMD Intro}
\begin{xca}
Please read the Introduction of Weapons of Math Destruction, by the
data scientist Cathy O'Neil (Links to an external site.)Links to an
external site., then give the following questions your best attempt.
Bring your answers to class for discussion.
\begin{enumerate}
\item According to O'Neil, what is a \textquotedbl weapon of math destruction\textquotedbl ?
\begin{enumerate}
\item a model that manages our lives and tends to reproduce social inequities
or inherit prejudice from its programmers/source datasets
\end{enumerate}
\item What \textquotedbl paradox\textquotedbl{} does O'Neil note about
the standard of evidence WMDs are held to as opposed to humans? How
might this paradox lead to social injustice?
\begin{enumerate}
\item They're black-boxed and need provide less evidence to justify profiling
a person than that person must provide to exonerate themselves
\end{enumerate}
\item According to O'Neil, in what ways do WMDs differ from the \textquotedbl effective
use{[}s{]} of statistics\textquotedbl{} in Google ad testing?
\begin{enumerate}
\item not enough controls/numbers to compare against
\end{enumerate}
\item What \textquotedbl payoff\textquotedbl{} do the companies who make
WMDs receive for their effort? According to O'Neil, how does this
\textquotedbl payoff\textquotedbl{} lead to a \textquotedbl dangerous
confusion\textquotedbl ?
\begin{enumerate}
\item they get lots of money and think it's because they're on the right
track
\end{enumerate}
\item What question(s) do you have about data science, statistical models,
and WMDs after reading the introduction to O'Neil's book?
\end{enumerate}
\end{xca}


\section{2018 State of the Numbers Worksheet}
\begin{xca}
The Twitter account for the 45th President of the United States of
America, @realDonaldTrump, is a consistent source of analyzable numerical
statements. Excerpts appear below. For each, get skeptical by doing
the following steps: 
\begin{itemize}
\item Decide what point the President is trying to make by citing these
numbers. 
\item Before pulling out an electronic device, brainstorm a list of questions
(3-5) you\textquoteright d like to know about the numbers\textendash things
that would help you understand if the numbers are big or small, and
whether or not the underlying data actually supports the President\textquoteright s
argument. 
\item Get online and try to answer your questions. 
\item Rate the President\textquoteright s claims on two scales 
\begin{itemize}
\item How true is the statement? (0 = completely false, 4 = completely true) 
\item How misleading is the statement? (0 = not at all, 4 = totally misleading)
(In other words, when you look at the underlying data, does it support
the President\textquoteright s point or not?)
\end{itemize}
\end{itemize}
\end{xca}

\begin{enumerate}
\item \textquotedblleft 90\% of media coverage of my Administration is negative,
despite the tremendously positive results we are achieving, it\textquoteright s
no surprise that confidence in the media is at an all time low.\textquotedblright{}
(July 29, 2018)
\begin{enumerate}
\item \href{https://www.washingtonpost.com/blogs/erik-wemple/wp/2017/09/12/study-91-percent-of-recent-network-trump-coverage-has-been-negative/?utm_term=.ca88176c0b03}{True},
but Trump actively courts media attention and ratings (announcing
the Arpaio pardoning as Harvey approached Texas because ``the ratings
would be far higher''
\item \href{https://www.washingtonpost.com/news/the-fix/wp/2018/06/01/the-case-for-even-tougher-media-coverage-of-trump/?utm_term=.4e9fe3786df5}{False};
``Trump appears to be cherry-picking figures when he claims 90-plus
percent of his media coverage is negative. Studies by Pew and Shorenstein
have found negativity rates in that range \textemdash{} but only among
news reports with clear tones. Both organizations have found that
about one-third of reports are neutral, meaning the true frequency
of negative reporting is considerably lower.''
\item Very misleading IMO: No other President in recent history has so courted
media coverage, going so far as to break democratic norms partially
to keep the camera on himself.
\end{enumerate}
\item \textquotedblleft When Trump visited the island territory last October,
OFFICIALS told him in a briefing 16 PEOPLE had died from Maria.\textquotedblright{}
The Washington Post. This was long AFTER the hurricane took place.
Over many months it went to 64 PEOPLE. Then, like magic, \textquotedblleft 3000
PEOPLE KILLED.\textquotedblright{} They hired GWU Research to tell
them how many people had died in Puerto Rico (how would they not know
this?). This method was never done with previous hurricanes because
other jurisdictions know how many people were killed. FIFTY TIMES
LAST ORIGINAL NUMBER - NO WAY! (September 14, 2018) 
\begin{enumerate}
\item \href{http://www.chicagotribune.com/news/sns-bc-us--trump-puerto-rico-20180914-story.html}{[source]}
Trump falsely accused Democrats on Thursday of inflating the Puerto
Rican toll to make him \textquotedbl look as bad as possible.\textquotedbl{}
He said just six to 18 people had been reported dead when he visited
two weeks after the October 2017 storm and suggested that many had
been added later \textquotedbl if a person died for any reason, like
old age.\textquotedbl{}
\item When Trump visited Puerto Rico, the death toll at the time was indeed
16 people. The number was later raised to 64, but the government then
commissioned an independent study to determine how many died because
of post-storm conditions. That study \textemdash{} conducted by the
Milken Institute School of Public Health at George Washington University
\textemdash{} estimated 2,975 deaths.
\item The deaths fell in two categories: direct and indirect. Direct deaths
include such fatalities as drownings in a storm surge or being crushed
in a wind-toppled building. Indirect deaths are harder to count because
they can include such things as heart attacks, electrocutions from
downed power lines and failure to receive dialysis because the power
is out \textemdash{} and those kinds of fatalities can happen after
a storm has ended but while an area is struggling to restore electricity,
clean water and other health and safety services.
\item Dr. Carlos Santos-Burgoa \textemdash{} the lead researcher on the
study and a well-known expert in global health, particularly Latin
America \textemdash{} told The Associated Press that the initial figure
of 64 deaths reflected only people whose death certificates cited
the storm. He said the latest figure was more accurate and stressed
that every death in the six months following the storm was not attributed
to the hurricane.
\end{enumerate}
\item \textquotedblleft Wow, highest Poll Numbers in the history of the
Republican Party. That includes Honest Abe Lincoln and Ronald Reagan.
There must be something wrong, please recheck that poll!\textquotedblright{}
(July 29, 2018) 
\begin{enumerate}
\item But the way Trump phrased his tweet was problematic. One concern involves
polling at the time of Honest Abe \textemdash{} to be precise, the
lack thereof. The other involves higher approval ratings of recent
Republican presidents. There were no scientific opinion polls in Lincoln\textquoteright s
day.
\item Unscientific straw polls date to at least 1824, with \textquotedbl informal
trial heat tallies\textquotedbl{} \textquotedbl taken in scattered
taverns, militia offices, and public meetings,\textquotedbl{} D. Sunshine
Hillygus, a Duke University political scientist, has written.
\item But the polls of Lincoln\textquoteright s time are not considered
scientific.
\item The idea of scientific polling came into its own after a disastrous
1936 poll by Literary Digest, a popular magazine. The magazine\textquoteright s
poll was based on responses from millions of Americans, but the data
it gathered had a significantly different demographic balance than
the electorate as a whole. This lack of scientific rigor led the poll
to incorrectly call the 1936 presidential election, opening the door
to survey methods championed by George Gallup and others.
\item Today, reputable pollsters use a better system: random sampling to
choose phone numbers from across the country, which they then call
to ask standardized, carefully worded questions. The assumption is
that a truly random sample will reproduce the whole nation in microcosm
to within a few percentage points of accuracy (the familiar \textquotedbl margin
of error\textquotedbl ).
\end{enumerate}
\begin{example}
Suppose that we sell your company several crates of oranges each week.
You examine a sample of oranges from each crate to determine the quality
of our oranges. It is easy to inspect a few oranges from the top of
each crate, but these oranges may not be representative of the entire
crate. Those on the bottom are more often damaged in shipment. If
we were less than honest, we might make sure that the rotten oranges
are packed on the bottom, with some good ones on top for you to inspect.
If you sample from the top, your sample results are again biased\textemdash the
sample oranges are systematically better than the population they
are supposed to represent.
\end{example}

\begin{defn}
The design of a statistical study is \textbf{biased} if it systematically
favors certain outcomes.

Selection of whichever individuals are easiest to reach is called
\textbf{convenience sampling}. {[}oranges{]}

A \textbf{voluntary response sample} chooses itself by responding
to a general appeal. Call-in or online opinion polls are examples
of voluntary response samples.

Convenience samples and voluntary response samples are often biased.
\end{defn}

\item \textquotedblleft Farmers have been on a downward trend for 15 years.
The price of soybeans has fallen 50\% since 5 years before the Election.
A big reason is bad (terrible) Trade Deals with other countries. They
put on massive Tariffs and Barriers. Canada charges 275\% on Dairy.
Farmers will WIN!\textquotedblright{} (July 20, 2018) 
\begin{enumerate}
\item {[}slide{]} Soybean futures soared to all-time highs in 2012 -{}-
five years before Trump was elected -{}- as a drought ravaged Midwest
farms and crimped production. In the seasons since then, farmers in
U.S. and Brazil, the top exporters, have reaped several back-to-back
bumper crops and prices have slumped.
\end{enumerate}
\item \textquotedblleft About 1-in-5 {[}undocumented{]} adults crossing
the United States-Mexico border with toddlers under the age of five
are either criminals, not the child\textquoteright s parent, or present
some other danger to the child...\textquotedblright{} (retweeted Dan
Scavino, Jr., July 14, 2018) 
\begin{enumerate}
\item {[}slide{]} crime rates have gone down by 36\% since 1980 as immigration,
documented and undocumented, has risen by $118\%$.
\item There is, however, one study that backs the president\textquoteright s
claim. John Lott, president of the Crime Prevention Research Center,
looked at data on prisoners in Arizona state prison between the beginning
of 1985 and June 2017 and concluded that \textquotedblleft undocumented
immigrants are at least 146\% more likely to be convicted of crime
than other Arizonans.\textquotedblright{} They also tend to commit
more serious crimes, and have significantly higher rates for such
crimes as murder, manslaughter, sexual assault and armed robbery,
Lott concluded, and are more likely to be gang members. Conversely,
Lott found that legal immigrants \textquotedblleft were extremely
law-abiding,\textquotedblright{} committing crimes at a lower rate
than native-born residents.
\item Although Lott says his study is unique because \textquotedblleft for
the first time\textquotedblright{} he was able to differentiate between
immigrants in the country legally and illegally, that claim was contested
by Nowrasteh of the Cato Institute. Nowrasteh argues Lott\textquoteright s
study contains a \textquotedblleft fatal flaw\textquotedblright{}
in its assumption that it was able to \textquotedblleft identify illegal
immigrants\textquotedblright{} from the data. The Washington Post
Fact Checker did a deep dive on the arguments and counterarguments
about the validity of the study.
\item \textquotedblleft The overall picture of immigrants and crime remains
confused due to a lack of good data and contrary information,\textquotedblright{}
Steven Camarota and Jessica Vaughan of the Center for Immigration
Studies, a group that advocates low immigration, wrote in 2009.
\end{enumerate}
\item \textquotedblleft Consumer Sentiment hit its highest level in 17 years
this year. Sentiment fell 11\% in 2015, an Obama year, and rose 16\%
since the Election, \#TrumpTime!\textquotedblright{} (September 19,
2018)
\begin{enumerate}
\item \href{https://www.bloomberg.com/news/articles/2017-10-31/u-s-consumer-confidence-index-rises-to-highest-level-since-2000}{True}:
U.S. consumer confidence rose more than expected in October to the
highest in almost 17 years as Americans grew more confident about
the economy and job market, according to figures Tuesday from the
New York-based Conference Board.
\end{enumerate}
\end{enumerate}

\end{document}
