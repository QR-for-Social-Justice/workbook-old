%% LyX 2.3.6.2 created this file.  For more info, see http://www.lyx.org/.
%% Do not edit unless you really know what you are doing.
\documentclass[oneside,english]{amsart}
\usepackage[T1]{fontenc}
\usepackage{geometry}
\geometry{verbose,tmargin=1in,bmargin=1in,lmargin=1in,rmargin=1in}
\setcounter{secnumdepth}{2}
\setcounter{tocdepth}{2}
\usepackage{amstext}
\usepackage{amsthm}
\usepackage{graphicx}

\makeatletter

%%%%%%%%%%%%%%%%%%%%%%%%%%%%%% LyX specific LaTeX commands.
%% Because html converters don't know tabularnewline
\providecommand{\tabularnewline}{\\}

%%%%%%%%%%%%%%%%%%%%%%%%%%%%%% Textclass specific LaTeX commands.
\numberwithin{equation}{section}
\numberwithin{figure}{section}
\theoremstyle{plain}
\newtheorem{thm}{\protect\theoremname}
\theoremstyle{definition}
\newtheorem{xca}[thm]{\protect\exercisename}
\theoremstyle{definition}
\newtheorem{example}[thm]{\protect\examplename}
\theoremstyle{definition}
\newtheorem*{example*}{\protect\examplename}
\theoremstyle{definition}
\newtheorem{defn}[thm]{\protect\definitionname}

\makeatother

\usepackage{babel}
\providecommand{\definitionname}{Definition}
\providecommand{\examplename}{Example}
\providecommand{\exercisename}{Exercise}
\providecommand{\theoremname}{Theorem}

\begin{document}
\title{WCSAM 206 Day 4 - Affine Cipher Encryption; Intro to Systems of Congruences}
\maketitle

\section{Last Time}
\begin{itemize}
\item Usually, if we're thinking about integers modulo $n$, we only want
to worry about the integers $0,1,2,\dots,n-1$. For instance, $12\equiv2$
mod $5$, so we think about $2$ mod 5 instead of 12.
\end{itemize}
We have two main ways to reduce $a$ mod $n$:
\begin{enumerate}
\item If $a\geq0$, we can replace $a$ with its remainder when it's divided
by $n$. Continuing Example 1, $7\equiv7$ mod 21 since $7\div21=0$
with remainder 7 and $14\equiv2$ mod 13 since $14\div3=4$ with remainder
2.
\item We can add or subtract multiple copies of $n$ since $n\equiv0$ mod
$n$, which is especially helpful for $a<0$. For example, $-7\equiv3$
mod 5 since $-7+2(5)=3$.
\end{enumerate}
\begin{itemize}
\item What does it mean if $n\mid(a-b)$? This is equivalent to saying $\frac{a-b}{n}$
is a whole number, for example $4$. But if $\frac{a-b}{n}=4$, then
$a-b=4n$, so $a=b+4n$. But $n$ is our clock size, so we can add
and subtract $4n$ without changing what time it is. Hence $a$ and
$b$ represent the same time on an $n$-hour clock if and only if
$n\mid(a-b)$, so our definitions are equivalent!
\end{itemize}
\begin{xca}
{[}slide{]} Use the letter-number chart at the top of p30 of your
Coursepack to:
\begin{enumerate}
\item Reduce {[}i.e.. find the time on an $n$-hour clock{]} the following
numbers mod $n$:
\begin{enumerate}
\item $184\mod27$
\begin{enumerate}
\item We have that $184\div27=6R22$, so $184\equiv22\mod27$.
\end{enumerate}
\item $12\mod3$
\begin{enumerate}
\item Since $12/3=4R0$, $12\equiv0\mod3$.
\end{enumerate}
\item $0\mod17$
\begin{enumerate}
\item $0$ mod any positive number is $0$ because $0/\text{positive number}=0R0$!
\end{enumerate}
\item $3\mod1$.
\begin{enumerate}
\item Any whole number mod $1$ is $0$ because $x/1=xR0$ for any whole
number $x$!
\end{enumerate}
\end{enumerate}
\end{enumerate}
\end{xca}

\begin{example}
Now, say we want to find something like $1080+108\mod108$. One way
of interpreting this is ``if we start at $0:00$, go forward $1080$
hours, then go forward $108$ hours, what time is it on a $108$-hour
clock?
\begin{itemize}
\item One way of solving this problem is by adding $1080+108=1188$, then
reducing $1188\mod108$.
\item Can anyone think of an easier way?
\begin{itemize}
\item Notice that going forward $1080$ hours brings us back to $0:00$,
since $1080\equiv0\mod108$.
\item Notice also that going forward $108$ hours on a $108$-hour clock
doesn't change the time. So we're still at $0$:$00$!
\item Thus, $1080+108\equiv0\mod108$.
\end{itemize}
\item We've just discovered the following important fact about modular arithmetic:
\begin{itemize}
\item if we're trying to reduce $a+b\mod n$, we can first reduce $a$,
then $b$, then add the results.
\end{itemize}
\end{itemize}
\end{example}

\begin{xca}
What about multiplication? That is, can we say that
\[
1080(108)\equiv0(0)=0\mod108?
\]
Think about the clock interpretation of $1080\times108$ on a $108$-hour
clock. What about of time are we going forward by, and how many times
are we going forward this amount? Then reduce the following:
\begin{enumerate}
\item $90+15\mod15$

Since $90=15\times6$, we have
\[
90+15\equiv15\times6+15\equiv0\times6+0=0\mod15.
\]

\item $90\times15\mod15$

Similarly, we have~$90\times15\equiv0\times0=0\mod15$.
\item $1155\times444\mod2$

Since odd numbers are congruent to $1$ and even numbers are congruent
to $0$ mod $2$, we have
\[
1155\times444\equiv1\times0=0\mod2.
\]
In particular, this says that the result of $1155\times444$ is even,
which may not be surprising.
\end{enumerate}
\end{xca}

\begin{itemize}
\item Now, we note that modular arithmetic gives us a method of encrypting/decrypting
messages using shift ciphers without having to count backwards/forwards
in the alphabet.
\item Also, modular arithmetic inspires a new cipher, noting that the graph
of the shift cipher $C\equiv P+k\mod26$ is a line with slope $1$,
but there's no obvious reason to restrict to slope $1$. 
\item The \textbf{affine cipher} encrypts a plaintext letter by first multiplying
it by some \textbf{multiplicative key} $m$, then adding some \textbf{additive
key $k$ }to the result.
\end{itemize}
\begin{example*}
Consider the affine cipher where the plaintext letters are first multiplied
by $3$ and then $1$ is added. Encrypt the phrase ``assassinate
the king now'' using this affine cipher.

We can still use a table like we did for the shift cipher:

\begin{tabular}{|c|c|c|c|c|c|c|c|c|c|c|c|}
\hline 
Plaintext letter & A & S & S & A & S & S & I & N & A & T & E\tabularnewline
\hline 
\hline 
Plaintext number & 0 & 18 & 18 & 0 & 18 & 18 & 8 & 13 & 0 & 19 & 4\tabularnewline
\hline 
$3\times$ plaintext number $+1$ (mod 26) & 1 & $-17\equiv9$ & 9 & 1 & 9 & 9 & 25 & $40\equiv14$ & 1 & $-20\equiv6$ & ...finish this yourselves.\tabularnewline
\hline 
Ciphertext letter & B &  &  &  &  &  &  &  &  &  & \tabularnewline
\hline 
Plaintext letter & T & H & E & K & I & N & G & N & O & W & \tabularnewline
\hline 
Plain number & 19 & 7 & 4 & 10 & 8 & 13 & 6 & 13 & 14 & 22 & \tabularnewline
\hline 
$3\times$ Plain number $+1$ &  &  &  &  &  &  &  &  &  &  & \tabularnewline
\hline 
Cipher  &  &  &  &  &  &  &  &  &  &  & \tabularnewline
\hline 
\end{tabular}

To write this cipher in equation form, we can call the resulting ciphertext
$C$. Then, this cipher is given by
\[
C\equiv3P+1\text{ mod 26}.
\]
\end{example*}
In general, the mathematical equation for the shift cipher with shift
$k$ is given by
\[
C\equiv P+k\text{ mod 26}
\]
and the equation for the affine cipher is given by
\[
C\equiv mP+k\text{ mod 26}(\ast)
\]
where $m$ is called the \textbf{multiplicative key} and \textbf{$k$
}is called the \textbf{additive key}.

Suppose you're given some plaintext. Remember that the Caesar or simple
shift cipher takes each plaintext letter and adds some constant amount
$k$ to it. The \textbf{affine cipher} first multiplies the plaintext
letter by a certain amount $m$, then adds a constant amount $k$. 

\includegraphics{\string"Old Cryptography Notes/pasted43\string".png}
\begin{xca}
{[}Worksheet{]} Affine Cipher 1-2 (Coursepack p110)
\end{xca}

~
\begin{xca}
\textbf{(Reading Question) }Complete the following exercises, writing
out your reasoning step-by-step as instructed in the specifications
for student work for this course. Then turn in your answers in class
on the date below.
\begin{enumerate}
\item Encrypt the following using the affine cipher $C\equiv mP+k\mod26$,
the standard alphabet, and the specified keys. 
\begin{enumerate}
\item \textquotedbl never say never\textquotedbl , $m=4,k=0$
\item \textquotedbl stop hammer time\textquotedbl , $m=5,k=2$ 
\item \textquotedbl supercalifragilisticexpialidocious\textquotedbl ,
$m=13,k=1$ {[}Hint: you can save yourself a lot of work by noticing
a pattern! Think about what happens when the letter being encrypted
is odd vs. even.{]} 
\end{enumerate}
\item Computers can only understand numbers (which are written in 0s and
1s), not letters. The computer language ASCII represents all commonly-used
written characters as numbers between 1 and 256. Do you think we can
encrypt ASCII messages using the affine cipher $C\equiv8P\mod256$?
Why or why not?
\end{enumerate}
\end{xca}

\begin{example}
But now let's say that we're trying to crack the affine cipher. Let's
take this ciphertext: \textbf{hikctyhghykaezztgaayrhbggrvgdyfg} (https://pastebin.com/8L4vCZQT)
and try to find $m$ and $k$.
\begin{itemize}
\item The most common letters are g and h, which appear six and four times,
respectively. This suggests that the ciphertext G and H may correspond
to plaintext e and t. Let's write the previous statement in number
form:
\item Ciphertext 6 corresponds to plaintext 4, and ciphertext 7 corresponds
to plaintext 19.
\item Let's look at the equation $(\ast)$ above. When $C=G=6$, then $P=e=4$
and the equation becomes
\[
6\equiv4m+k.
\]
When $C=H=7$, then $P=t=19$ and the equation becomes
\[
7\equiv19m+k.
\]
\item Together, these equations give us a \textbf{system of congruences}
\begin{eqnarray*}
4m+k & \equiv & 6\\
19m+k & \equiv & 7
\end{eqnarray*}
(both mod 26). But we don't know yet how to solve systems of congruences!
\end{itemize}
\end{example}

\begin{xca}
If this were a system of \emph{equations}, not congruences, what would
your steps be?
\end{xca}


\section*{Solving Systems of Congruences}
\begin{itemize}
\item We'll start with some more basic facts and build up to solving systems
of congruences.
\end{itemize}
\begin{defn}
Recall that a number is \textbf{prime} if the only integers that divide
it are $1$ and itself. We'll use $p$ and $q$ to represent prime
numbers. \textbf{Composite numbers} are integers greater than one
that are not prime.
\end{defn}

\begin{itemize}
\item You'll remember that every positive whole number has a unique \textbf{prime
factorization}, i.e. a way to write it as a product of prime numbers.
This fact is called the \textbf{Fundamental Theorem of Arithmetic.}
\end{itemize}
\includegraphics{\string"Old Cryptography Notes/pasted39\string".png}
\begin{xca}
{[}if time{]} Python module work
\end{xca}


\end{document}
